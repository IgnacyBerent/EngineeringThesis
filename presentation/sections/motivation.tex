\section{Motivation}
\begin{frame}{Biological Context}
  \begin{columns}[T] % [T] aligns columns at the top
    \begin{column}{0.48\textwidth}
      \begin{figure}
        \centering
        \includegraphics[height=0.75\textheight]{figs/ans_anatomy.png}
        \vspace{-0.7cm}
        \caption{Autonomic Nervous System (ANS) Structure}
        \vspace{-0.4cm}
        {\tiny Source: \cite{ans_anatomy}}
    \end{figure}
  \end{column}

    \begin{column}{0.48\textwidth}
      \begin{block}{The Baroreflex Mechanism}
              \begin{itemize}
                  \item \textbf{Homeostatic Function:} One of the most critical regulatory systems governing the Autonomic Nervous System (ANS) and cardiovascular stability, controling arterial blood pressure (ABP). \cite{heartmetronome}.
                  \item \textbf{Negative Feedback:} Baroreceptors sense ABP fluctuations and modulate the ANS to adjust the heart period (HP) \cite{heartmetronome}.
                  \item \textbf{Complexity:} A healthy system exhibits high variability; a "loss of complexity" is often a marker of aging or pathology (like early stage of cardiovascular disease) \cite{lossofcomplexity}.
              \end{itemize}
          \end{block}
    \end{column}
  \end{columns}
\end{frame}

\begin{frame}{Baroreflex and it's Sensitivity (BRS)}
  \begin{columns}[T] 
        \begin{column}{0.48\textwidth}
        \begin{block}{Baroreflex Sensitivity (BRS)}
            \begin{itemize}
              \item \textbf{Definition:} The quantitative relationship between the SAP stimulus (input) and the HP reflex response (output) \cite{baroreflexcurve}.
              \item \textbf{Closed-Loop Nature:} Bidirectional interaction (Neural: SAP $\to$ HP vs. Mechanical: HP $\to$ SAP) complicates linear analysis \cite{javorka2017causal}.
              \item \textbf{Confounding Factors:} Respiration ("respiratory gate") and $CO_2$ levels can modulate or override the baroreflex response \cite{russo2017physiological, Uryga_2024}.
          \end{itemize}
        \end{block}
        \end{column}

        \begin{column}{0.48\textwidth}
            \begin{figure}
                \centering
                \includegraphics[width=0.8\textwidth]{figs/baroreflex-curve.png}
                \vspace{-0.4cm}
                \caption{Synthetically generated baroreflex curve of healthy subject based on \cite{baroreflexcurve}}
            \end{figure}
        \end{column}
        
    \end{columns}
\end{frame}

\begin{frame}{BRS Estimation Techniques}
\begin{enumerate}
        \item \textbf{Pharmacological method} -- not suitable for experiments including large cohort of subjects
        \item \textbf{Time Series Analysis} -- don't include the directionality of the relation
        \begin{enumerate}
        \item frequency / time cross correlation functions
        \item frequency / time coherence functions
        \item frequency spectrum analysis
        \end{enumerate}
        \item \textbf{Mutual Information} - handle nonlinearity, but not directional
        \item \textbf{Granger Causality} -- directional, but linear
        \item \textbf{Transfer Entropy (TE)} -- directional, handle nonlinear and dynamic relationships, model free with the right implementation
        \item \textbf{Conditional Joint Transfer Entropy (CJTE)} -- expands TE with conditioning, to eliminate biases \cite{mehta2018directional}, \cite{shahsavari2020estimating}
    \end{enumerate}
\end{frame}
