\section{Implementation}
\begin{frame}{Implementation Chalanges}
  \begin{block}{Chalanges}
    \begin{itemize}
        \item \textbf{Empirical Uncertainty:} The true probability distribution of transition states is unknown and must be estimated from finite data.
        \item \textbf{Computational Complexity:} The original formulation requires iterating over all possible state combinations, leading to exponential scaling.
        \item \textbf{Methodological Sensitivity:} The choice of estimation method dictates how data is handeled inside TE, but also very result of the calculation \cite{genccaga2018transfer}, \cite{rozo2021benchmarking}, \cite{lee2012transfer}.
    \end{itemize}
  \end{block}

  \begin{block}{Proposed Solution}
    \begin{itemize}
      \item Implement from scratch Darbellay-Vajda Partitioning (DVP) algorithm for probability estimation \cite{dvp}, \cite{rozo2021benchmarking}, \cite{lee2012transfer}.
      \item Do it for both TE and CJTE 
      \item Make it general for any number of dimensions
    \end{itemize}
  \end{block}
\end{frame}

\begin{frame}{Code Diagram}
\begin{figure}
        \centering
        \includegraphics[height=0.82\textheight]{figs/code_diagram_v2.png}
        \vspace{-0.4cm}
        \caption{Diagram representing the complete workflow of the implemented code}
    \end{figure}
\end{frame}

\begin{frame}{Example flow - Initial Signals}
\begin{figure}
        \centering
        \includegraphics[width=0.9\textwidth]{figs/implementation/Initial Signals.png}
\end{figure}
\end{frame}

\begin{frame}{Example flow - Dimensional Embedding}
\begin{figure}
        \centering
        \includegraphics[width=0.9\textwidth]{figs/implementation/Embedded Signals.png}
\end{figure}
\end{frame}

\begin{frame}{Example flow - Ordinar Ranking}
\begin{figure}
        \centering
        \includegraphics[width=0.9\textwidth]{figs/implementation/Ordinary Rank Transformed Signals.png}
\end{figure}
\end{frame}

\begin{frame}{Example Flow: DVP and TE Calculation}
  \begin{columns}[T]
        \begin{column}{0.48\textwidth}
            \begin{figure}
                \centering
                \vspace{-0.6cm}
                \includegraphics[width=1.1\textwidth]{figs/implementation/Darbellay-Vajda Adaptive Partitioning (3D).png}
            \end{figure}
          \end{column}

        \begin{column}{0.48\textwidth}
            \begin{exampleblock}{Subspace Mapping}
                For each $i$-th partition, state probabilities are substituted with box counts:
                \begin{equation}
                \small
                TE_{Y\to X}^{d} = \sum p(a_i) \log_2\left(\frac{p(a_i)p(b_i)}{p(c_i)p(d_i)}\right)
                \end{equation}
                \scriptsize
                Mapping from thesis Section 4.3:
                \begin{itemize}
                    \item $p(a_i) \to$ Joint ($X_t, X_{t-1}^d, Y_{t-1}^d$)
                    \item $p(b_i) \to$ Past ($X_{t-1}^d$)
                    \item $p(c_i) \to$ ($X_t, X_{t-1}^d$)
                    \item $p(d_i) \to$ ($X_{t-1}^d, Y_{t-1}^d$)
                \end{itemize}
                \textbf{Estimation:} $p(a_i) = \frac{n_{a,i}}{N}$ \cite{dvp}
            \end{exampleblock}
        \end{column}
    \end{columns}
\end{frame}


