\chapter{Introduction}\label{ch:intro}

\section{Aim}\label{sec:aim}
The aim of this work is to conduct a comparative analysis of algorithms for transfer entropy (TE) and conditional joint transfer entropy (CJTE), with consideration of a modulating variable. The analysis will assess the impact of analytical parameters used in entropy estimation on the results of TE and CJTE in both simulated and physiological signals.


\section{Scope}\label{sec:scope}
\begin{enumerate}
    \item Literature review and analysis of algorithms (methods) within the transfer entropy group, focusing on key aspects such as signal duration, calculation window, sampling rate, type of physiological signals, study protocol, and analytical parameters
    \item Implementation of selected algorithms for transfer entropy (TE) and conditional joint transfer entropy (CJTE)
    \item Analysis of the impact of analytical parameters on TE and CJTE results in simulated signals
    \item Analysis of the impact of analytical parameters on TE and CJTE results in physiological signals
    \item Statistical analysis of the results to evaluate the usefulness of TE and CJTE algorithms for assessing relationship between signals and parameters related to cerebral autoregulation and the autonomic nervous system, considering the modulating variable
\end{enumerate}

\newpage
\section{Motivation}\label{sec:motivation}
A human body is an extremly complex biological machinery build from numerous mutually dependent mechanisms. 
Their main objective is to maintain physiological homeostasis, which can be described using metrics such as pH, body temperature, or heart rate.
To cope with a constantly changing environment, the organism must continously adjust its physiological parameters through leveraging mechanisms, that operate across multiple time scales - from miliseconds to minutes, hours, or even days \cite{heartmetronome}.

One of the key regulatory systems is heart-brain interaction governed by Autonomic Nervous System (ANS). 
The ANS is responsible for involuntary function of most internal organs and consist of two major divisions: the Sympathetic Nervous System (SNS), and Parasympathetic Nervous System (PNS).
Although they act in opposing directions, the shifts in their relative balance is crucial for cardiovascular stability and homeostasis.
The SNS is responsible for "fight or flight" response, increasing heart rate and cardiac output, whereas PNS mediates "rest and digest" state, slowing down the heart and facilitating recovery.
Importantly, heart rate regulation emerges from nonlinear and dynamic interactions between blood pressure, respiration, and autonomic neural activity \cite{reyes2013utility};\cite{hirsch1981respiratory};\cite{hirsch1996role};\cite{mccraty2009coherent}).

In this context, the heart rate variability (HRV) is recognized as a non-invasive measure of neurocardiac function. 
A healthy heart is not expected to function as a perfect metronome, rather natural fluctuations in RR intervals reflect a valid physilogical responsiveness \cite{heartmetronome}.
Numerous studies have shown that reduced HRV and diminished complexity are associated with autonomic dysfunction and increased risk of health disorders (\cite{tsuji1996impact};\cite{carney2001depression};\cite{agelink2002relationship};\cite{giardino2004combined};\cite{lehrer2004biofeedback};\cite{cohen2006power}).
This phenomenon aligns with the loss-of-complexity theory, which suggests that aging and patological states often suppress inherent self-regulatory mechanisms, reducing variability and ability to adapt to stress \cite{lossofcomplexity}.
However, large-scale measurement of (HRV) is challenging because it requires electrocardiography (ECG).
For this reason, baroreflex sensitivity (BRS) can be used as an alternative, offering the benefit of being acquirable from commonly used, non-invasive arterial blood pressure (ABP) measurement alone \cite{la1995baroreflex}.

However, the simple time domain measurements, or spectral analysis techniques, which are based on linear mathematics, are not sufficient in assesing the complex nature of such nonlinear and dynamic systems (\cite{heartmetronome};\cite{lossofcomplexity}).
Based on concepts of Information Theory by Shannon, new techniques of accessing casual relationships in complex systems have been introduced with big potential in analysis of the physiological data, such as covered in this work Transfer Entropy (TE) (\cite{Shannon1948}; \cite{Schreiber2000};
\cite{genccaga2018transfer};\cite{wen2023kendall}).
