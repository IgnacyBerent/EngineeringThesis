\chapter{Introduction}\label{ch:intro}

\section{Aim}\label{sec:aim}
The aim of this work is to conduct a comparative analysis of algorithms for transfer entropy (TE) and conditional joint transfer entropy (CJTE), with consideration of a modulating variable. The analysis will assess the impact of analytical parameters used in entropy estimation on the results of TE and CJTE in both simulated and physiological signals.


\section{Scope}\label{sec:scope}
\begin{enumerate}
    \item Literature review and analysis of algorithms (methods) within the transfer entropy group, focusing on key aspects such as signal duration, calculation window, sampling rate, type of physiological signals, study protocol, and analytical parameters
    \item Implementation of selected algorithms for transfer entropy (TE) and conditional joint transfer entropy (CJTE)
    \item Analysis of the impact of analytical parameters on TE and CJTE results in simulated signals
    \item Analysis of the impact of analytical parameters on TE and CJTE results in physiological signals
    \item Statistical analysis of the results to evaluate the usefulness of TE and CJTE algorithms for assessing relationship between signals and parameters related to cerebral autoregulation and the autonomic nervous system, considering the modulating variable
\end{enumerate}

\section{Motivation}\label{sec:motiv}
\subsection{Nonlinear metrics in signal analysis}
The human body is a source of many different types of signals caused by different physiological mechanisms that exist to maintain homeostasis.
These signals often capture different patterns on different time scales, creating complex, nonlinear interactions \cite{Bogli2024}.
A good example of such a system is, analyzed later in this study, the baroreflex. 

\subsection{Entropy approach applications}

\subsection{Literature overview}
