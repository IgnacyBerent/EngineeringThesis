\chapter{Introduction}\label{ch:intro}

\section{Aim}\label{sec:aim}
The aim of this work is to conduct a comparative analysis of algorithms for transfer entropy (TE) and conditional joint transfer entropy (CJTE), with consideration of a modulating variable. The analysis will assess the impact of analytical parameters used in entropy estimation on the results of TE and CJTE in both simulated and physiological signals.


\section{Scope}\label{sec:scope}
\begin{enumerate}
    \item Literature review and analysis of algorithms (methods) within the transfer entropy group, focusing on key aspects such as signal duration, calculation window, sampling rate, type of physiological signals, study protocol, and analytical parameters
    \item Implementation of selected algorithms for transfer entropy (TE) and conditional joint transfer entropy (CJTE)
    \item Analysis of the impact of analytical parameters on TE and CJTE results in simulated signals
    \item Analysis of the impact of analytical parameters on TE and CJTE results in physiological signals
    \item Statistical analysis of the results to evaluate the usefulness of TE and CJTE algorithms for assessing relationship between signals and parameters related to cerebral autoregulation and the autonomic nervous system, considering the modulating variable
\end{enumerate}

\newpage
\section{Motivation}\label{sec:motiv}
%TODO: 1, 1.5 strony
% Heart rate is not a metromone
% Ogólna prawda: Organiozm , układy autoregulujące homeostaza, reagują w zależności od różnych bodźców na skali milisekund
% Autonomiczny ukłąd Nerwowy - odpowiada za pracę prawie wszytskich organów wciele, łuk odruchowy jest złożony  a daynamika jest nielioniowa, więc do opisu trzeba stosować zaawansowane nieliniowe metryki
% Przy zaburzonej homestozie (przy patologiach) następuje przytłumienie mechanizmów autoregulujących powodująć spadek entropi (teoria dekompleksyfikacji) (przy asystoli / migotanie przedsionkow moze sie podniesc)
% Użycie zaawansowanych metod pozwlana na pełne przeanalizowanie zmienności
% Poco chcemy coś warunkować, wpływ innych czynników fizjologicznych na HRV
% Czego nam brakuje?
Our human body is an extremly complicated machinery build from many dependent on each other mechanisms. 
The aim of these mechanisms is to maintain so called homeostasis, which is a "steady-state".
We are using different metrics to describe this state like ph, body temperature, heart rate.
Depending on the constantly changing environment the human body has to adjust itself using different mechanisms that work on different time scales from miliseconds to minutes, or even hours or days.
This mechanisms often demonstrate complex nonlinear dynamics, entagled often in various types of feedback loops. 
In result studing this systems is not an easy task, that require using more profound techniques, to be able to analysie relationships that take place. \cite{heartmetronome}

One of the most important system in our body is Autonomic Nervous System (ANS). 
It is responsible for involuntary functions in our organism, and has two major divisions - Sympathetic Nervous System (SNS) and Parasympathetic Nervous System (PNS).
The two systems work in the opposite directions to mainain homeostatis.
SNS is responsible for "fight or flight" response that causes increased heart rate, in contrast PNS tends to slow it and activates "rest and digest" response.
For this reason each measurement of heart rate is the result of actual dynamic balance between those two components, and because of that we should not expect our heart to work as a perfect metronome.

Many studies suggest that variability in heart rate (HRV), which is a reflection of dynamic relationships in our body, carries information from both short-term and long-term memory functions that take place in ANS.
Low HRV can be used as predictor of future heatlh problems, and is linked to many autonomic dysfuncions. 
Because of that studing it and researching influence of different physiological factors can be crucial in developing modern techniques in the disease treatment.

