\chapter{Conclusions}
The presented implementation of both TE and CJTE, using the DVP estimation technique allowed for a detailed characterization of information flow in both synthetic and physiological systems.
Additionally, the provided framework allows for more in depth analysis of these algorithms.

By comparing synthetic benchmarks with physiological measurements, this study highlights that TE and CJTE are powerful tools for detecting directionality and nonlinearity. 
While TE is an effective standalone tool in bivariate systems, CJTE provides superior insights in multivariate physiological environments where multiple regulatory factors interact. 
However, a key limitation identified is the potential for misleading results (synergistic inflation), when a strong common driver is present alongside weak direct interaction. 
Therefore, CJTE should not be used in isolation but rather complemented by TE to ensure the integrity of the causal interpretation.

Future work should investigate bias-correction techniques or the systematic optimization of hyperparameter tuning, to further refine the partitioning process and improve the specificity of the CJTE metric.
Additionally, longer measurements on larger cohorts, including pathological populations or varying age groups, would be beneficial to generalize these results and refine the proposed BRS estimation technique.

