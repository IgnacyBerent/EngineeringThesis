\chapter*{Abstract}
Information theory-based methods are increasingly popular for analyzing causal relationships. 
Unlike classical methods such as spectral analysis or Granger causality, these techniques can detect directional dependencies in dynamic, nonlinear systems.

This study compares Transfer Entropy (TE) and its advanced variant, Conditional Joint Transfer Entropy (CJTE). 
Because these techniques require transition state probability estimation, results vary significantly by implementation. 
The Darbellay-Vajda Partitioning (DVP) algorithm was selected for its reliability and model-free approach.

To evaluate these functions, three synthetic signal types were generated: a bivariate linear model, a bivariate nonlinear model, and a trivariate linear model with a common driver. 
For physiological application, the baroreflex system was analyzed. 
As a primary regulator of the autonomic nervous system, the baroreflex maintains homeostasis by a negative feedback loop that stabilizes heart period (HP) despite shifts in systolic arterial pressure (SAP). 
Baroreflex sensitivity (BRS) quantifies how HP fluctuates in response to SAP changes. 
Notably, the chemoreflex effect, driven by end-tidal carbon dioxide (ETCO$_2$) levels, often acts as a confounding factor in controlled breathing protocols, introducing bias in BRS estimation.

Physiological signals were retrospectively analysed, from 32 healthy volunteers (14 males, 18 females, median age: 22 years) during the \textit{AUTOMATIC} project at Wrocław University of Science and Technology. 
Measurements were physician-supervised and approved by the Bioethics Commission (KB-179/2023/N). 
Participants underwent four 5-minute stages: breathing at rest, and at 6, 10, and 15 breaths per minute. 
Arterial blood pressure (ABP) and ETCO$_2$ were recorded noninvasively. 
SAP and HP were then derived from the ABP signal.

Results showed that TE effectively detects both linear and nonlinear causalities in synthetic signals and remains resilient to mild noise. 
Findings suggest that longer signal windows of at least $500$ to $1,000$ samples improve TE estimation precision.

A comparison of TE and CJTE was conducted on the trivariate linear system and the baroreflex, both involving common driver characteristics. 
TE falsely identified the confounding force mediated by $Z$ as the driving force of the $X \to Y$ relationship. 
However, CJTE with DVP showed limitations: in cases of strong driving forces and weak direct coupling, synergistic inflation occurred, overestimating information transfer values. 
This was also observed in BRS estimations during breathing at rest, 10 bpm, and 15 bpm. 
Nevertheless, CJTE successfully reduced the ETCO$_2$ effect at the 6 bpm peak.

The implementation of both TE and CJTE using the DVP algorithm, alongside the development of a systematic evaluation framework, enabled a robust comparison of these techniques.
To address current limitations, CJTE should be used alongside TE rather than as a standalone method to ensure a comprehensive system explanation, until better, bias-reduction methods are found. 
Future research with larger cohorts and longer recordings will offer more generalized findings.

\paragraph*{Keywords}
Transfer Entropy, Conditional Joint Transfer Entropy, Baroreflex Sensitivity, Darbellay-Vajda Partitioning, Autonomic Nervous System, Information Theory, Nonlinear Dynamics

