\chapter*{Abstract (ENG)}
Information theory-based methods are increasingly popular for analyzing causal relationships. 
Unlike classical methods such as spectral analysis or Granger causality, these techniques can detect directional dependencies in dynamic, nonlinear systems.

This study compares Transfer Entropy (TE) and its advanced variant, Conditional Joint Transfer Entropy (CJTE). 
Because these techniques require transition state probability estimation, results vary significantly by implementation. 
The Darbellay-Vajda Partitioning (DVP) algorithm was selected for its reliability and model-free approach.

To evaluate these functions, three synthetic signal types were generated: a bivariate linear model, a bivariate nonlinear model, and a trivariate linear model with a common driver. 
For physiological application, the baroreflex system was analyzed. 
As a primary regulator of the autonomic nervous system, the baroreflex maintains homeostasis by a negative feedback loop that stabilizes heart period (HP) despite shifts in systolic arterial pressure (SAP). 
Baroreflex sensitivity (BRS) quantifies how HP fluctuates in response to SAP changes. 
Notably, the chemoreflex effect, driven by end-tidal carbon dioxide (ETCO$_2$) levels, often acts as a confounding factor in controlled breathing protocols, introducing bias in BRS estimation.

Physiological signals were retrospectively analysed, from 32 healthy volunteers (14 males, 18 females, median age: 22 years) during the \textit{AUTOMATIC} project at Wrocław University of Science and Technology. 
Measurements were physician-supervised and approved by the Bioethics Commission (KB-179/2023/N). 
Participants underwent four 5-minute stages: breathing at rest, and at 6, 10, and 15 breaths per minute. 
Arterial blood pressure (ABP) and ETCO$_2$ were recorded noninvasively. 
SAP and HP were then derived from the ABP signal.

Results showed that TE effectively detects both linear and nonlinear causalities in synthetic signals and remains resilient to mild noise. 
Findings suggest that longer signal windows of at least $500$ to $1,000$ samples improve TE estimation precision.

A comparison of TE and CJTE was conducted on the trivariate linear system and the baroreflex, both involving common driver characteristics. 
TE falsely identified the confounding force mediated by $Z$ as the driving force of the $X \to Y$ relationship. 
However, CJTE with DVP showed limitations: in cases of strong driving forces and weak direct coupling, synergistic inflation occurred, overestimating information transfer values. 
This was also observed in BRS estimations during breathing at rest, 10 bpm, and 15 bpm. 
Nevertheless, CJTE successfully reduced the ETCO$_2$ effect at the 6 bpm peak.

The implementation of both TE and CJTE using the DVP algorithm, alongside the development of a systematic evaluation framework, enabled a robust comparison of these techniques.
To address current limitations, CJTE should be used alongside TE rather than as a standalone method to ensure a comprehensive system explanation, until better, bias-reduction methods are found. 
Future research with larger cohorts and longer recordings will offer more generalized findings.

\paragraph*{Keywords:}
Transfer Entropy, Conditional Joint Transfer Entropy, Baroreflex Sensitivity, Darbellay-Vajda Partitioning, Autonomic Nervous System, Information Theory, Nonlinear Dynamics

\chapter*{Abstrakt (PL)}
Metody oparte na teorii informacji zyskują coraz większą popularność w analizie związków przyczynowych. 
W przeciwieństwie do klasycznych metod, takich jak analiza spektralna czy przyczynowość Grangera, techniki te pozwalają na wykrywanie kierunkowych zależności w dynamicznych układach nieliniowych.

W niniejszej pracy porównano Entropię Przenoszenia (Transfer Entropy – TE) oraz jej zaawansowany wariant – Warunkową Łączną Entropię Przenoszenia (Conditional Joint Transfer Entropy – CJTE). 
Ponieważ techniki te wymagają oszacowania prawdopodobieństwa stanów przejścia, wyniki różnią się znacząco w zależności od zastosowanej implementacji. 
Wybrano algorytm partycjonowania Darbellay-Vajda (DVP) ze względu na jego niezawodność i podejście nieparametryczne (model-free).

W celu ewaluacji tych funkcji wygenerowano trzy rodzaje sygnałów syntetycznych: dwuwymiarowy model liniowy, dwuwymiarowy model nieliniowy oraz trójwymiarowy model liniowy ze wspólnym źródłem (common driver). 
W ramach zastosowań fizjologicznych analizie poddano układ barorefleksu. 
Jako główny regulator autonomicznego układu nerwowego, barorefleks utrzymuje homeostazę poprzez pętlę ujemnego sprzężenia zwrotnego, która stabilizuje czas trwania cyklu pracy serca (HP) pomimo zmian skurczowego ciśnienia tętniczego (SAP). 
Wrażliwość barorefleksu (BRS) określa ilościowo, jak zmienia się HP w odpowiedzi na zmiany SAP. 
Należy zauważyć, że efekt chemorefleksu, stymulowany poziomem dwutlenku węgla w końcowej fazie wydechu (ETCO2), często działa jako czynnik zakłócający w protokołach z kontrolowanym oddechem, wprowadzając błąd w oszacowaniu BRS.


Sygnały fizjologiczne 32 zdrowych ochotników (14 mężczyzn, 18 kobiet, mediana wieku: 22 lata) zebrane podczas projektu AUTOMATIC na Politechnice Wrocławskiej poddano analizie retrospektywnej. 
Pomiary były nadzorowane przez lekarza i zatwierdzone przez Komisję Bioetyczną (KB-179/2023/N). 
Uczestnicy przeszli cztery 5-minutowe etapy: oddech spoczynkowy oraz oddech z częstotliwością 6, 10 i 15 oddechów na minutę. 
Ciśnienie tętnicze krwi (ABP) oraz ETCO2 rejestrowano nieinwazyjnie. Wartości SAP i HP wyznaczono następnie z sygnału ABP.

Wyniki wykazały, że TE skutecznie wykrywa przyczynowość zarówno liniową, jak i nieliniową w sygnałach syntetycznych i pozostaje odporna na umiarkowany szum. 
Wyniki sugerują, że dłuższe okna sygnałowe (co najmniej $500$ do $1000$ próbek) poprawiają precyzję estymacji TE. 
Porównanie TE i CJTE przeprowadzono na trójwymiarowym układzie liniowym oraz na barorefleksie – w obu przypadkach występowała charakterystyka wspólnego źródła. 
TE błędnie zidentyfikowała wpływ czynnika zakłócającego (pośredniczonego przez Z) jako siłę napędową relacji $X\to Y$. 
Jednakże CJTE z algorytmem DVP wykazała ograniczenia: w przypadkach silnych sił wymuszających i słabego sprzężenia bezpośredniego dochodziło do synergistycznego zawyżenia wyników (synergistic inflation), co prowadziło do przeszacowania wartości transferu informacji. 
Zjawisko to zaobserwowano również w estymacjach BRS podczas oddechu spoczynkowego oraz przy 10 i 15 oddechach na minutę. 
Niemniej jednak, CJTE skutecznie zredukowała wpływ ETCO2 przy piku 6 oddechów na minutę.

Implementacja zarówno TE, jak i CJTE przy użyciu algorytmu DVP, wraz z opracowaniem systematycznych ram ewaluacji, umożliwiła rzetelne porównanie tych technik. 
Aby zaradzić obecnym ograniczeniom, CJTE powinna być stosowana równolegle z TE, a nie jako samodzielna metoda, co zapewni kompleksowe wyjaśnienie systemu do czasu znalezienia lepszych metod redukcji obciążeń (bias). 
Przyszłe badania na większych kohortach i dłuższych nagraniach pozwolą na uzyskanie bardziej uogólnionych wniosków.

\paragraph*{Słowa Kluczowe:}
Entropia Przenoszenia, Warunkowa Łączna Entropia Przenoszenia, Wrażliwość Barorefleksu, Partycjonowanie Darbellay-Vajda, Autonomiczny Układ Nerwowy, Teoria Informacji, Dynamika Nieliniowa.


