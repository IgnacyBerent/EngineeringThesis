\chapter{Materials \& Methods}\label{ch:matmeth}

\section{Synthetic signals}

\section{Physiological signals}\label{sec:physsig}

\subsection{Research group}\label{subsec:researchgr}
The data set with the physiological signals come from the study conducted for the AUTOMATIC project conducted at Wrocaw University of Wroclaw and Technology (WUST). 
It consists of 37 young healthy volunteers (21 males, 16 males, median age: 22 years, range 18-31 years). 
Everyone was instructed not to consume alcohol and caffeine 12 hours before the study and declared to be free of medication. 
Measurements were taken under the supervision of a profesional physician and were approved by the Comission of Bioethics (KB-179/2023/N) \citep{Uryga_2024}.

\subsection{Data acquisition}\label{subsec:dataacq}
The signals used are: arterial blood pressure (ABP), recorded noninvasively using a photoplethysmograph (Finometer MIDI) and end--tidal carbon dioxide (ETCO$_{2}$), recorded using a capnograph and used as modulating factor.

Measurements were made at room temperature with minimal external stimuli as in Figure \ref{fig:experiment-setup} and were carried out in four stages lasting 5 minutes each and performed one after another. 
In stage 1 subjects were breathing at their resting rate, and in stages 2,3,4 they were performing controlled breathing guided by metromone, respectively, at rates 6,10 and 15 breaths per minute (bpm).

For simplification, only signals of 6 and 15 bpm were taken under analysis. 

\begin{figure}[H]
    \centering
    \includegraphics[width=0.5\linewidth]{figs/experiment_setup.jpg}
    \caption{The experiment setup}
    \label{fig:experiment-setup}
\end{figure}

\subsection{Implementation}\label{sec:implementation}
Firstly both SAP and HP were derived from the ABP signal using follwoing formulas:
\begin{equation}\label{eq:sap_from_abp}
    SAP(i) = ABP(peak(i))\text{ } [mm/Hg]
\end{equation}
\begin{equation}\label{eq:rr_from_abp}
    RR(i) = \frac{peak(i+1) - peak(i)}{fs}\text{ } [s]
\end{equation}
\begin{equation}
    HP(i) = \frac{1}{RR(i)} \text{ } [s]
\end{equation}

Where $fs$ is a sampling frequency and $peak(i)$ is the index of the i-th peak in the abp signal found using the \textit{ppg\_clean} function in the neurokit2 python library \cite{neurokit2}. 
Visualization of how it works is shown in the Figure \ref{fig:peaks-finding}.
\begin{figure}[H]
    \centering
    \includegraphics[width=\linewidth]{figs/peaks_finding.png}
    \caption{Method of calculating SAP and HP from ABP signal}
    \label{fig:peaks-finding}
\end{figure}

%TODO: Opisać że dane byly wyczyszczone
%TODO: Opisać jak neurokit znajduje peaki + przykłady
