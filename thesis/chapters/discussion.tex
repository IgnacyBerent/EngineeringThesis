\chapter{Discussion}
The implementation of TE and CJTE using the DVP estimator was rigorously evaluated across five synthetic scenarios and validated through the physiological phenomenon of the baroreflex. 
While the synthetic signals were not intended to replicate biological data exactly, they successfully demonstrated the general dynamics and robustness of the DVP-based information-theoretic approach.

\section{Sensitivity to Coupling Strength and Nonlinearity}
The most critical requirement for any information flow estimator is the ability to detect varying degrees of coupling. 
The results in Subsection \ref{subsec:effect_of_linear_coupling_strength} confirm that TE accurately captures linear interactions within an AR(1) framework. 
Notably, as shown in Subsection \ref{subsec:effect_of_nonlinear_coupling_strength}, TE identifies increasing information flow in nonlinear systems without requiring prior model assumptions. 
This "model-free" nature is a significant advantage over traditional methods like Granger causality.

Furthermore, Subsubsection \ref{subsubsubsec:drive_detection} demonstrates that the strength of a common driving factor can be equally distinguished on both signals $X$ and $Y$. 
Crucially, TE remained statistically insignificant in directions opposing the natural flow. 
The residual low values observed in these counter-examples are attributed to the non-negativity constraint of the DVP estimator; since it estimates probability densities via partitioning, the resulting entropy values are inherently positive even in the absence of a true causal link.

\section{The Impact of Signal Length}
The analysis of signal length (Subsection \ref{subsec:effect_of_signal_length}) suggests that DVP accuracy improves with larger datasets. 
The findings indicate that while the physiological segments used in this thesis provided stable results, future studies could benefit from longer recording windows to minimize the estimation bias inherent in data-partitioning methods.

\section{The Impact of Noise}
Regarding signal quality, Subsection \ref{subsec:effect_of_noise} demonstrates that the DVP-based TE is remarkably robust to stochastic interference. 
Significant degradation in information flow detection only occurred at an SNR of 10 or lower, suggesting that the method is well-suited for clinical environments where physiological signals (like ECG or blood pressure) are often contaminated by high-frequency noise.

\section{The Confounding Effect of Common Drivers}
The synthetic trivariate model was designed to simulate the impact of a common driver $Z$ (representing ETCO$_2$) on the interaction between two variables $X$ and $Y$ (representing $SAP$ and $HP$). 
Results presented in Subsection \ref{subsec:effect_of_confounding_variable} reveal that while CJTE aims to isolate direct coupling, it does not completely neutralize the influence of the common drive. 
Both TE and CJTE values were found to increase alongside the driving strength $a_z$, a phenomenon that can be attributed to \textit{synergistic inflation} rather than direct information transfer.

As observed in Subsubsection \ref{subsubsubsec:synthetic_comparison_cjte-te}, in the absence of a direct relationship between $X$ and $Y$ ($a_x = 0$), CJTE yielded significantly higher values than standard TE. 
This indicates a synergistic interaction: when Z is known, the residual information in X, originating from the common drive, becomes more predictive of Y. 
This leads to an inflation of the CJTE despite the absence of a direct physical coupling.
This serves as a critical methodological caution: a higher CJTE value does not always imply a stronger physical connection if a potent third-party driver is unaccounted for.

\section{Physiological Interpretation and BRS}
The physiological application of these metrics confirmed several established cardiovascular phenomena. 
The observed peak in Baroreflex Sensitivity (BRS) at 6 breaths/min aligns with the resonance effect between Respiratory Sinus Arrhythmia (RSA) and the baroreflex control loop. 
Conversely, the attenuation at 15 breaths/min reflects the low-pass filtering characteristics of the Autonomic Nervous System (ANS), where the heart's response to rapid blood pressure fluctuations is dampened due to the finite response times of the autonomic branches \cite{heartmetronome}.

In the context of ETCO$_2$ (Subsection \ref{subsec:phys-etco-effect}), the TE results suggest that ETCO$_2$ exerts a relatively constant baseline influence on both $SAP$ and $HP$ across all tasks. 
A clear analogy can be drawn between the physiological results in Table \ref{subsec:phys-comparison-results} and the synthetic results in Table \ref{tab:comparison-trivariate-ax}. 
At breathing frequencies where direct baroreflex coupling is lower (10 and 15 breaths/min), the CJTE values appear slightly inflated by the persistent ETCO$_2$ drive. 

However, at the 6 breaths/min resonance frequency, the significant increase in both TE and CJTE confirms that a genuine and powerful \textit{direct coupling} ($a_x > 0$) has emerged. 
This suggests that the baroreflex is the primary regulatory force at this frequency. 
If hypoventilation had occurred specifically at 6 breaths/min, the information transfer from ETCO$_2$ to both $SAP$ and $HP$ would have spiked, causing the CJTE to "flatten" relative to the TE. 
Since the ETCO$_2$ transfer remained stable, the CJTE effectively validates the baroreflex as the dominant regulatory mechanism during slow-breathing resonance.

\section{Hyperparameter Tuning}
While the DVP implementation is generally considered model-free, three parameters significantly influence its performance: dimensional embedding ($d$), time delay ($\tau$), and the significance level ($\alpha$) for the uniformity test. 
Literature suggests that TE performance in higher-order synthetic relationships can be improved by adjusting $\tau$ \cite{rozo2021benchmarking}. 
Since most research defaults to standard values, the optimization of these parameters—particularly for CJTE—remains a promising area for future study.
