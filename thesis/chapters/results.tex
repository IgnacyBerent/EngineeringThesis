\chapter{Results}
Each analysis included calculation of the repeated measures analysis of variance (RM ANOVA), followed by the post hoc analysis using two-sided, pairwase, t-tests, with \textit{Bonferroni} $p$-value adjustments.
Comparisons of TE with CJTE were performed using paired, two-sided t-tests.

The following acronyms are used in the results tables, which are composed of the statistical calculation of the RM ANOVA (top panel) and the post hoc analysis (bottom panel):

\begin{itemize}
  \item \textbf{ddof1, ddof2}: Degrees of freedom for the RM ANOVA numerator and denominator, respectively.
  \item \textbf{F}: The F-statistic value from the RM ANOVA.
  \item \textbf{p-unc}: Uncorrected $p$-value (used in both ANOVA and post hoc tables).
  \item \textbf{$\eta^2_G$ (ng2)}: Generalized Eta Squared, representing the effect size for the RM ANOVA.
  \item \textbf{$\epsilon$ (eps)}: Epsilon, such as the Greenhouse–Geisser correction factor for sphericity violation.
  \item \textbf{T}: The T-statistic value from the pairwise t-tests.
  \item \textbf{dof}: Degrees of freedom for the pairwise t-tests.
  \item \textbf{p-corr}: Bonferroni-corrected $p$-value from the post hoc analysis.
\end{itemize}

\section{Synthetic Signals}

\subsection{Effect of Linear Coupling Strength}\label{subsec:effect_of_linear_coupling_strength}

\subsubsection{Effect on $TE_{X \to Y}$}
Acording to Figure \ref{fig:openloop-linear-a-along} and results of statistical analysis in the Table \ref{tab:openloop-linear-a-along} there is significant increase in TE accompanying increasing $a$.

Post hoc analysis show that the significant increase wasn't only shown between $a=0$ and $a=0.1$, while mean TE value for $a=0$ is close to 0, what is expected since it would suggest that the coupling strength is near 0 too.

\begin{figure}[H]
    \centering
    \includegraphics[width=0.9\linewidth]{figs/results/synthetic/Varying a Linear Bivariate te_x->y.png}
    \caption{TE results for varying linear coupling strength ($a$) in synthetic signals, along the real information flow direction ($X \to Y$)}
    \label{fig:openloop-linear-a-along}
\end{figure}

\begin{table}[H]
\centering
\caption{Statistical analysis values for the Transfer Entropy results along the direction $X \to Y$ (Figure \ref{fig:openloop-linear-a-along})}
\begin{tabular}{llllll}
\toprule
ddof1 & ddof2 & F & p-unc & ng2 & eps \\
\midrule
$3$ & $297$ & $236.26$ & $< 0.001$ & $0.578$ & $0.807$ \\
\bottomrule
\end{tabular}
\vspace{0.5em}
\begin{tabular}{llllll}
\toprule
A & B & T & dof & p-unc & p-corr \\
\midrule
a=0 & a=0.1 & $-2.626$ & $99$ & $0.010$ & $0.060$ \\
a=0 & a=0.25 & $-9.603$ & $99$ & $< 0.001$ & $< 0.001$ \\
a=0 & a=0.5 & $-21.17$ & $99$ & $< 0.001$ & $< 0.001$ \\
a=0.1 & a=0.25 & $-8.003$ & $99$ & $< 0.001$ & $< 0.001$ \\
a=0.1 & a=0.5 & $-18.116$ & $99$ & $< 0.001$ & $< 0.001$ \\
a=0.25 & a=0.5 & $-14.376$ & $99$ & $< 0.001$ & $< 0.001$ \\
\bottomrule
\end{tabular}
\label{tab:openloop-linear-a-along}
\end{table}

\subsubsection{Effect on $TE_{Y \to X}$}
Conversely, when measuring TE in the direction against the real information flow ($Y \to X$), no statistically significant difference is observed within increasing $a$ values, as shown in Table \ref{tab:openloop-linear-a-against} and Figure \ref{fig:openloop-linear-a-against}.

\begin{figure}[H]
    \centering
    \includegraphics[width=0.9\linewidth]{figs/results/synthetic/Varying a Linear Bivariate te_y->x.png}
    \caption{TE results for varying linear coupling strength ($a$) in synthetic signals, against the real information flow direction ($Y \to X$)}
    \label{fig:openloop-linear-a-against}
\end{figure}

\begin{table}[H]
\centering
\caption{Statistics analysis values for the figure \ref{fig:openloop-linear-a-against}}
\begin{tabular}{llllll}
\toprule
ddof1 & ddof2 & F & p-unc & ng2 & eps \\
\midrule
$3$ & $297$ & $1.263$ & $0.287$ & $0.006$ & $0.835$ \\
\bottomrule
\end{tabular}
\vspace{0.5em}
\begin{tabular}{llllll}
\toprule
A & B & T & dof & p-unc & p-corr \\
\midrule
a=0 & a=0.1 & $-0.818$ & $99$ & $0.416$ & $1.000$ \\
a=0 & a=0.25 & $0.121$ & $99$ & $0.904$ & $1.000$ \\
a=0 & a=0.5 & $-1.418$ & $99$ & $0.159$ & $0.956$ \\
a=0.1 & a=0.25 & $0.77$ & $99$ & $0.443$ & $1.000$ \\
a=0.1 & a=0.5 & $-0.996$ & $99$ & $0.322$ & $1.000$ \\
a=0.25 & a=0.5 & $-1.699$ & $99$ & $0.092$ & $0.554$ \\
\bottomrule
\end{tabular}
\label{tab:openloop-linear-a-against}
\end{table}

\subsection{Effect of Signal Length}\label{subsec:effect_of_signal_length}

\subsubsection{Effect on $TE_{X \to Y}$}
Post hoc analysis present in Table \ref{tab:openloop-linear-l-along} show that length of the signal has a statistically significant impact on TE evaluation in the direction of the coupling.
In addition, observing Figure \ref{fig:openloop-linear-l-along} allows to notice that variance is decreasing with increasing length (from $0.41$ to $0.18$).

\begin{figure}[H]
    \centering
    \includegraphics[width=0.9\linewidth]{figs/results/synthetic/Varying Length Linear Bivariate te_x->y.png}
    \caption{TE results for varying signal length in a linearly coupled synthetic signals, along the real information flow direction ($X \to Y$)}
    \label{fig:openloop-linear-l-along}
\end{figure}

\begin{table}[H]
\centering
\caption{Statistical analysis values for the TE results along the direction $X \to Y$ (Figure \ref{fig:openloop-linear-l-along})}
\begin{tabular}{llllll}
\toprule
ddof1 & ddof2 & F & p-unc & ng2 & eps \\
\midrule
$3$ & $297$ & $34.382$ & $< 0.001$ & $0.188$ & $0.788$ \\
\bottomrule
\end{tabular}
\vspace{0.5em}
\begin{tabular}{llllll}
\toprule
A & B & T & dof & p-unc & p-corr \\
\midrule
Length=100 & Length=200 & $-3.753$ & $99$ & $< 0.001$ & $< 0.01$ \\
Length=100 & Length=500 & $-6.657$ & $99$ & $< 0.001$ & $< 0.001$ \\
Length=100 & Length=1000 & $-9.193$ & $99$ & $< 0.001$ & $< 0.001$ \\
Length=200 & Length=500 & $-3.225$ & $99$ & $< 0.01$ & $0.010$ \\
Length=200 & Length=1000 & $-5.18$ & $99$ & $< 0.001$ & $< 0.001$ \\
Length=500 & Length=1000 & $-3.181$ & $99$ & $< 0.01$ & $0.012$ \\
\bottomrule
\end{tabular}
\label{tab:openloop-linear-l-along}
\end{table}

\subsubsection{Effect on $TE_{Y \to X}$}
On the other hand, no significant effect of signal length in the opposite direction is found, as Table \ref{tab:openloop-linear-l-against}, and Figure \ref{fig:openloop-linear-l-against} suggest.
However decreased variance with increasing length is also visible (from $0.081$ to $0.030$).

\begin{figure}[H]
    \centering
    \includegraphics[width=0.9\linewidth]{figs/results/synthetic/Varying Length Linear Bivariate te_y->x.png}
    \caption{TE results for varying signal length in a linearly coupled synthetic signals, against the real information flow direction ($Y \to X$)}
    \label{fig:openloop-linear-l-against}
\end{figure}

\begin{table}[H]
\centering
\caption{Statistical analysis values for the TE results against the direction $Y \to X$ (Figure \ref{fig:openloop-linear-l-against})}
\begin{tabular}{llllll}
\toprule
ddof1 & ddof2 & F & p-unc & ng2 & eps \\
\midrule
$3$ & $297$ & $1.088$ & $0.354$ & $0.008$ & $0.8$ \\
\bottomrule
\end{tabular}
\vspace{0.5em}
\begin{tabular}{llllll}
\toprule
A & B & T & dof & p-unc & p-corr \\
\midrule
Length=100 & Length=200 & $0.488$ & $99$ & $0.627$ & $1.000$ \\
Length=100 & Length=500 & $0.908$ & $99$ & $0.366$ & $1.000$ \\
Length=100 & Length=1000 & $-0.653$ & $99$ & $0.515$ & $1.000$ \\
Length=200 & Length=500 & $0.458$ & $99$ & $0.648$ & $1.000$ \\
Length=200 & Length=1000 & $-1.389$ & $99$ & $0.168$ & $1.000$ \\
Length=500 & Length=1000 & $-2.566$ & $99$ & $0.012$ & $0.071$ \\
\bottomrule
\end{tabular}
\label{tab:openloop-linear-l-against}
\end{table}

\subsection{Effect of Noise}\label{subsec:effect_of_noise}

\subsubsection{Effect on $TE_{X \to Y}$}
The results for the TE measured along the true causal direction ($X \to Y$) are presented in Figure \ref{fig:openloop-linear-e-along} and Table \ref{tab:openloop-linear-e-along}.
Post hoc analysis show that results for the biggest ammount of noice (SNR=$10$), are only significant different, while mean of this difference is only $17\%$ smaller than mean for no noise.

\begin{figure}[H]
    \centering
    \includegraphics[width=0.9\linewidth]{figs/results/synthetic/Varying SNR Linear Bivariate te_x->y.png}
    \caption{TE results for different SNR values in a linearly coupled synthetic signal, along the real information flow direction ($X \to Y$)}
    \label{fig:openloop-linear-e-along}
\end{figure}

\begin{table}[H]
\centering
\caption{Statistical analysis values for the TE results along the direction $X \to Y$ (Figure \ref{fig:openloop-linear-e-along})}
\begin{tabular}{llllll}
\toprule
ddof1 & ddof2 & F & p-unc & ng2 & eps \\
\midrule
$3$ & $297$ & $12.8$ & $< 0.001$ & $0.052$ & $0.727$ \\
\bottomrule
\end{tabular}
\vspace{0.5em}
\begin{tabular}{llllll}
\toprule
A & B & T & dof & p-unc & p-corr \\
\midrule
SNR=None & SNR=30 & $1.293$ & $99$ & $0.199$ & $1.000$ \\
SNR=None & SNR=20 & $1.991$ & $99$ & $0.049$ & $0.296$ \\
SNR=None & SNR=10 & $4.594$ & $99$ & $< 0.001$ & $< 0.001$ \\
SNR=30 & SNR=20 & $1.339$ & $99$ & $0.184$ & $1.000$ \\
SNR=30 & SNR=10 & $4.182$ & $99$ & $< 0.001$ & $< 0.001$ \\
SNR=20 & SNR=10 & $3.71$ & $99$ & $< 0.001$ & $< 0.01$ \\
\bottomrule
\end{tabular}
\label{tab:openloop-linear-e-along}
\end{table}

% \subsubsection{Effect on $TE_{Y \to X}$}
%
%
% \begin{figure}[H]
%     \centering
%     \includegraphics[width=0.9\linewidth]{figs/results/synthetic/Varying SNR Linear Bivariate te_y->x.png}
%     \caption{TE results for different SNR values in a linearly coupled synthetic signal, against the real information flow direction ($Y \to X$)}
%     \label{fig:openloop-linear-e-against}
% \end{figure}
%
% \begin{table}[H]
% \centering
% \caption{Statistical analysis values for the TE results against the direction $Y \to X$ (Figure \ref{fig:openloop-linear-e-against})}
% \begin{tabular}{llllll}
% \toprule
% ddof1 & ddof2 & F & p-unc & ng2 & eps \\
% \midrule
% $3$ & $297$ & $1.127$ & $0.338$ & $0.006$ & $0.84$ \\
% \bottomrule
% \end{tabular}
% \vspace{0.5em}
% \begin{tabular}{llllll}
% \toprule
% A & B & T & dof & p-unc & p-corr \\
% \midrule
% SNR=None & SNR=30 & $0.52$ & $99$ & $0.604$ & $1.000$ \\
% SNR=None & SNR=20 & $0.479$ & $99$ & $0.633$ & $1.000$ \\
% SNR=None & SNR=10 & $1.554$ & $99$ & $0.123$ & $0.740$ \\
% SNR=30 & SNR=20 & $0.107$ & $99$ & $0.915$ & $1.000$ \\
% SNR=30 & SNR=10 & $1.239$ & $99$ & $0.218$ & $1.000$ \\
% SNR=20 & SNR=10 & $1.246$ & $99$ & $0.216$ & $1.000$ \\
% \bottomrule
% \end{tabular}
% \label{tab:openloop-linear-e-against}
% \end{table}

\subsection{Effect of Nonlinear Coupling Strength}\label{subsec:effect_of_nonlinear_coupling_strength}

\subsubsection{Effect on $TE_{X \to Y}$}
Results from Table \ref{tab:openloop-nonlinear-gamma-along}, present on Figure \ref{fig:openloop-nonlinear-gamma-along} show significant steep increase of TE with increasing nonlinear coupling strength $b$ for any combination of groups.

\begin{figure}[H]
    \centering
    \includegraphics[width=0.9\linewidth]{figs/results/synthetic/Varying b Nonlinear Bivariate te_x->y.png}
    \caption{TE results for varying nonlinear coupling strength ($b$) in synthetic signals, along the real information flow ($X \to Y$)}
    \label{fig:openloop-nonlinear-gamma-along}
\end{figure}

\begin{table}[H]
\centering
\caption{Statistical analysis values for the Transfer Entropy results along the direction $X \to Y$ (Figure \ref{fig:openloop-nonlinear-gamma-along})}
\begin{tabular}{llllll}
\toprule
ddof1 & ddof2 & F & p-unc & ng2 & eps \\
\midrule
$3$ & $297$ & $7572.812$ & $< 0.001$ & $0.98$ & $0.647$ \\
\bottomrule
\end{tabular}
\vspace{0.5em}
\begin{tabular}{llllll}
\toprule
A & B & T & dof & p-unc & p-corr \\
\midrule
b=0 & b=0.1 & $-11.33$ & $99$ & $< 0.001$ & $< 0.001$ \\
b=0 & b=0.25 & $-51.515$ & $99$ & $< 0.001$ & $< 0.001$ \\
b=0 & b=0.5 & $-121.297$ & $99$ & $< 0.001$ & $< 0.001$ \\
b=0.1 & b=0.25 & $-46.631$ & $99$ & $< 0.001$ & $< 0.001$ \\
b=0.1 & b=0.5 & $-113.037$ & $99$ & $< 0.001$ & $< 0.001$ \\
b=0.25 & b=0.5 & $-76.093$ & $99$ & $< 0.001$ & $< 0.001$ \\
\bottomrule
\end{tabular}
\label{tab:openloop-nonlinear-gamma-along}
\end{table}

\subsubsection{Effect on $TE_{Y \to X}$}
The expected non-present difference between groups (Table \ref{tab:openloop-nonlinear-gamma-against}) is visible for the direction against the information flow, with visible on Figure \ref{fig:openloop-nonlinear-gamma-against} mean TE close to $0$.

\begin{figure}[H]
    \centering
    \includegraphics[width=0.9\linewidth]{figs/results/synthetic/Varying b Nonlinear Bivariate te_y->x.png}
    \caption{TE results for varying nonlinear coupling strength ($b$) in synthetic signals, against the real information flow ($X \to Y$)}
    \label{fig:openloop-nonlinear-gamma-against}
\end{figure}

\begin{table}[H]
\centering
\caption{Statistical analysis values for the Transfer Entropy results against the direction $Y \to X$ (Figure \ref{fig:openloop-nonlinear-gamma-against})}
\begin{tabular}{llllll}
\toprule
ddof1 & ddof2 & F & p-unc & ng2 & eps \\
\midrule
$3$ & $297$ & $0.559$ & $0.643$ & $0.004$ & $0.89$ \\
\bottomrule
\end{tabular}
\vspace{0.5em}
\begin{tabular}{llllll}
\toprule
A & B & T & dof & p-unc & p-corr \\
\midrule
b=0 & b=0.1 & $0.383$ & $99$ & $0.703$ & $1.000$ \\
b=0 & b=0.25 & $1.163$ & $99$ & $0.248$ & $1.000$ \\
b=0 & b=0.5 & $0.485$ & $99$ & $0.629$ & $1.000$ \\
b=0.1 & b=0.25 & $1.003$ & $99$ & $0.318$ & $1.000$ \\
b=0.1 & b=0.5 & $0.217$ & $99$ & $0.829$ & $1.000$ \\
b=0.25 & b=0.5 & $-0.77$ & $99$ & $0.443$ & $1.000$ \\
\bottomrule
\end{tabular}
\label{tab:openloop-nonlinear-gamma-against}
\end{table}

\subsection{Effect of a Confounding Variable}\label{subsec:effect_of_confounding_variable}

\subsubsection{Effect on $TE_{X \to Y}$}
Results from Tables \ref{tab:trivariate-te-x-y-az}, \ref{tab:trivariate-te-x-y-ax} visible on Figures \ref{fig:trivariate-te-x-y-az}, \ref{fig:trivariate-te-x-y-ax} show that both $a_z$, and $a_x$ impact TE results.

\begin{figure}[H]
    \centering
    \includegraphics[width=0.9\linewidth]{figs/results/synthetic/Varying az Linear Trivariate te_x->y.png}
    \caption{TE results for varying linear common drive strength ($a_z$) in synthetic trivariate system, along the real information flow ($X \to Y$)}
    \label{fig:trivariate-te-x-y-az}
\end{figure}

\begin{figure}[H]
    \centering
    \includegraphics[width=0.9\linewidth]{figs/results/synthetic/Varying ax Linear Trivariate te_x->y.png}
    \caption{TE results for varying linear coupling strength between $X$ and $Y$ ($a_x$) in synthetic trivariate system along the real information flow ($X \to Y$)}
    \label{fig:trivariate-te-x-y-ax}
\end{figure}

\begin{table}[H]
\centering
\caption{Statistical analysis values for the Standard TE results, $X \to Y$ (Figure \ref{fig:trivariate-te-x-y-az})}
\begin{tabular}{llllll}
\toprule
ddof1 & ddof2 & F & p-unc & ng2 & eps \\
\midrule
$3$ & $297$ & $213.898$ & $< 0.001$ & $0.525$ & $0.918$ \\
\bottomrule
\end{tabular}
\vspace{0.5em}
\begin{tabular}{llllll}
\toprule
A & B & T & dof & p-unc & p-corr \\
\midrule
az=0 & az=0.1 & $-3.383$ & $99$ & $< 0.01$ & $< 0.01$ \\
az=0 & az=0.25 & $-12.798$ & $99$ & $< 0.001$ & $< 0.001$ \\
az=0 & az=0.5 & $-21.672$ & $99$ & $< 0.001$ & $< 0.001$ \\
az=0.1 & az=0.25 & $-8.685$ & $99$ & $< 0.001$ & $< 0.001$ \\
az=0.1 & az=0.5 & $-19.375$ & $99$ & $< 0.001$ & $< 0.001$ \\
az=0.25 & az=0.5 & $-10.038$ & $99$ & $< 0.001$ & $< 0.001$ \\
\bottomrule
\end{tabular}
\label{tab:trivariate-te-x-y-az}
\end{table}

\begin{table}[H]
\centering
\caption{Statistical analysis values for the Standard TE results, $X \to Y$ (Figure \ref{fig:trivariate-te-x-y-ax})}
\begin{tabular}{llllll}
\toprule
ddof1 & ddof2 & F & p-unc & ng2 & eps \\
\midrule
$3$ & $297$ & $351.976$ & $< 0.001$ & $0.613$ & $0.85$ \\
\bottomrule
\end{tabular}
\vspace{0.5em}
\begin{tabular}{llllll}
\toprule
A & B & T & dof & p-unc & p-corr \\
\midrule
ax=0 & ax=0.1 & $-6.252$ & $99$ & $< 0.001$ & $< 0.001$ \\
ax=0 & ax=0.25 & $-15.904$ & $99$ & $< 0.001$ & $< 0.001$ \\
ax=0 & ax=0.5 & $-26.18$ & $99$ & $< 0.001$ & $< 0.001$ \\
ax=0.1 & ax=0.25 & $-10.606$ & $99$ & $< 0.001$ & $< 0.001$ \\
ax=0.1 & ax=0.5 & $-22.707$ & $99$ & $< 0.001$ & $< 0.001$ \\
ax=0.25 & ax=0.5 & $-14.089$ & $99$ & $< 0.001$ & $< 0.001$ \\
\bottomrule
\end{tabular}
\label{tab:trivariate-te-x-y-ax}
\end{table}

\subsubsection{Effect on $TE_{Y \to X}$}
In the opposite direction only $a_z$ causes TE to significantly differ among it's range, what is shown in Tables \ref{tab:trivariate-te-y-x-az}, \ref{tab:trivariate-te-y-x-ax} and Figures \ref{fig:trivariate-te-y-x-az}, \ref{fig:trivariate-te-y-x-ax}

\begin{figure}[H]
    \centering
    \includegraphics[width=0.9\linewidth]{figs/results/synthetic/Varying az Linear Trivariate te_y->x.png}
    \caption{TE results for varying linear common drive strength ($a_z$) in synthetic trivariate system, against the real information flow ($Y \to X$)}
    \label{fig:trivariate-te-y-x-az}
\end{figure}

\begin{figure}[H]
    \centering
    \includegraphics[width=0.9\linewidth]{figs/results/synthetic/Varying ax Linear Trivariate te_y->x.png}
    \caption{TE results for varying linear coupling strength between $Y$ and $X$ ($a_x$) in synthetic trivariate system against the real information flow ($Y \to X$)}
    \label{fig:trivariate-te-y-x-ax}
\end{figure}

\begin{table}[H]
\centering
\caption{Statistical analysis values for the Standard TE results, $Y \to X$ (Figure \ref{fig:trivariate-te-y-x-az})}
\begin{tabular}{llllll}
\toprule
ddof1 & ddof2 & F & p-unc & ng2 & eps \\
\midrule
$3$ & $297$ & $128.168$ & $< 0.001$ & $0.47$ & $0.804$ \\
\bottomrule
\end{tabular}
\vspace{0.5em}
\begin{tabular}{llllll}
\toprule
A & B & T & dof & p-unc & p-corr \\
\midrule
az=0 & az=0.1 & $-1.275$ & $99$ & $0.205$ & $1.000$ \\
az=0 & az=0.25 & $-6.571$ & $99$ & $< 0.001$ & $< 0.001$ \\
az=0 & az=0.5 & $-16.763$ & $99$ & $< 0.001$ & $< 0.001$ \\
az=0.1 & az=0.25 & $-6.59$ & $99$ & $< 0.001$ & $< 0.001$ \\
az=0.1 & az=0.5 & $-15.578$ & $99$ & $< 0.001$ & $< 0.001$ \\
az=0.25 & az=0.5 & $-8.568$ & $99$ & $< 0.001$ & $< 0.001$ \\
\bottomrule
\end{tabular}
\label{tab:trivariate-te-y-x-az}
\end{table}

\begin{table}[H]
\centering
\caption{Statistical analysis values for the Standard TE results, $Y \to X$ (Figure \ref{fig:trivariate-te-y-x-ax})}
\begin{tabular}{llllll}
\toprule
ddof1 & ddof2 & F & p-unc & ng2 & eps \\
\midrule
$3$ & $297$ & $2.39$ & $0.069$ & $0.01$ & $0.926$ \\
\bottomrule
\end{tabular}
\vspace{0.5em}
\begin{tabular}{llllll}
\toprule
A & B & T & dof & p-unc & p-corr \\
\midrule
ax=0 & ax=0.1 & $-0.429$ & $99$ & $0.669$ & $1.000$ \\
ax=0 & ax=0.25 & $-1.504$ & $99$ & $0.136$ & $0.814$ \\
ax=0 & ax=0.5 & $-2.161$ & $99$ & $0.033$ & $0.199$ \\
ax=0.1 & ax=0.25 & $-1.4$ & $99$ & $0.165$ & $0.988$ \\
ax=0.1 & ax=0.5 & $-1.898$ & $99$ & $0.061$ & $0.363$ \\
ax=0.25 & ax=0.5 & $-0.721$ & $99$ & $0.473$ & $1.000$ \\
\bottomrule
\end{tabular}
\label{tab:trivariate-te-y-x-ax}
\end{table}

\subsubsection{Effect on $TE_{Z \to X}$ and $TE_{Z \to Y}$}\label{subsubsubsec:drive_detection}
There is visible on Figures \ref{fig:trivariate-combined-te-az}, \ref{fig:trivariate-combined-te-ax} a simillar statistically significant effect of driving force $a_z$ on TE, and no effect of caused by the $a_x$.

\begin{figure}[H]
    \centering
    \begin{subfigure}[b]{0.48\textwidth}
        \centering
        \includegraphics[width=\linewidth]{figs/results/synthetic/Varying az Linear Trivariate te_z->x.png}
        \caption{Direction $Z \to X$}
        \label{fig:trivariate-te-z-x-az}
    \end{subfigure}
    \begin{subfigure}[b]{0.48\textwidth}
        \centering
        \includegraphics[width=\linewidth]{figs/results/synthetic/Varying az Linear Trivariate te_z->y.png}
        \caption{Direction $Z \to Y$}
        \label{fig:trivariate-te-z-y-az}
    \end{subfigure}
    
    \caption{TE results for varying linear common drive strength ($a_z$) in the synthetic trivariate system for the driving directions (a) $Z \to X$ and (b) $Z \to Y$.}
    \label{fig:trivariate-combined-te-az}
\end{figure}

\begin{figure}[H]
    \centering
    \begin{subfigure}[b]{0.48\textwidth}
        \centering
        \includegraphics[width=\linewidth]{figs/results/synthetic/Varying ax Linear Trivariate te_z->x.png}
        \caption{Direction $Z \to X$}
        \label{fig:trivariate-te-z-x-ax}
    \end{subfigure}
    \begin{subfigure}[b]{0.48\textwidth}
        \centering
        \includegraphics[width=\linewidth]{figs/results/synthetic/Varying ax Linear Trivariate te_z->y.png}
        \caption{Direction $Z \to Y$}
        \label{fig:trivariate-te-z-y-ax}
    \end{subfigure}
    
    \caption{TE results for varying $Y \to X$ coupling strength ($a_x$) in the synthetic trivariate system for the driving directions (a) $Z \to X$ and (b) $Z \to Y$.}
    \label{fig:trivariate-combined-te-ax}
\end{figure}

\begin{table}[H]
\centering
\caption{Combined statistical results for varying $a_z$ for the Figure \ref{fig:trivariate-combined-te-az}}
\begin{tabular}{lllllll}
\toprule
Source & ddof1 & ddof2 & F & p-unc & ng2 & eps \\
\midrule
$TE_{Z \to X}$ & $3$ & $297$ & $584.331$ & $< 0.001$ & $0.763$ & $0.72$ \\
$TE_{Z \to Y}$ & $3$ & $297$ & $587.409$ & $< 0.001$ & $0.783$ & $0.862$ \\
\bottomrule
\end{tabular}
\vspace{0.5em}
\begin{tabular}{lllllll}
\toprule
Source & A & B & T & dof & p-unc & p-corr \\
\midrule
$TE_{Z \to X}$ & az=0 & az=0.1 & $-5.995$ & $99$ & $< 0.001$ & $< 0.001$ \\
$TE_{Z \to X}$ & az=0 & az=0.25 & $-14.547$ & $99$ & $< 0.001$ & $< 0.001$ \\
$TE_{Z \to X}$ & az=0 & az=0.5 & $-30.236$ & $99$ & $< 0.001$ & $< 0.001$ \\
$TE_{Z \to X}$ & az=0.1 & az=0.25 & $-13.333$ & $99$ & $< 0.001$ & $< 0.001$ \\
$TE_{Z \to X}$ & az=0.1 & az=0.5 & $-30.774$ & $99$ & $< 0.001$ & $< 0.001$ \\
$TE_{Z \to X}$ & az=0.25 & az=0.5 & $-22.888$ & $99$ & $< 0.001$ & $< 0.001$ \\
$TE_{Z \to Y}$ & az=0 & az=0.1 & $-6.15$ & $99$ & $< 0.001$ & $< 0.001$ \\
$TE_{Z \to Y}$ & az=0 & az=0.25 & $-17.418$ & $99$ & $< 0.001$ & $< 0.001$ \\
$TE_{Z \to Y}$ & az=0 & az=0.5 & $-33.383$ & $99$ & $< 0.001$ & $< 0.001$ \\
$TE_{Z \to Y}$ & az=0.1 & az=0.25 & $-13.808$ & $99$ & $< 0.001$ & $< 0.001$ \\
$TE_{Z \to Y}$ & az=0.1 & az=0.5 & $-31.85$ & $99$ & $< 0.001$ & $< 0.001$ \\
$TE_{Z \to Y}$ & az=0.25 & az=0.5 & $-19.749$ & $99$ & $< 0.001$ & $< 0.001$ \\
\bottomrule
\end{tabular}
\label{tab:trivariate-combined-te-az}
\end{table}

\begin{table}[H]
\centering
\caption{Combined statistical results for varying $a_x$ for the Figure \ref{fig:trivariate-combined-te-ax}}
\begin{tabular}{lllllll}
\toprule
Source & ddof1 & ddof2 & F & p-unc & ng2 & eps \\
\midrule
$TE_{Z \to X}$ & $3$ & $297$ & $0.0$ & $1.0$ & $None$ & $None$ \\
$TE_{Z \to Y}$ & $3$ & $297$ & $3.249$ & $0.022$ & $0.013$ & $0.865$ \\
\bottomrule
\end{tabular}
\vspace{0.5em}
\begin{tabular}{lllllll}
\toprule
Source & A & B & T & dof & p-unc & p-corr \\
\midrule
$TE_{Z \to X}$ & ax=0 & ax=0.1 & None & $99$ & $1.0$ & $1.0$ \\
$TE_{Z \to X}$ & ax=0 & ax=0.25 & None & $99$ & $1.0$ & $1.0$ \\
$TE_{Z \to X}$ & ax=0 & ax=0.5 & None & $99$ & $1.0$ & $1.0$ \\
$TE_{Z \to X}$ & ax=0.1 & ax=0.25 & None & $99$ & $1.0$ & $1.0$ \\
$TE_{Z \to X}$ & ax=0.1 & ax=0.5 & None & $99$ & $1.0$ & $1.0$ \\
$TE_{Z \to X}$ & ax=0.25 & ax=0.5 & None & $99$ & $1.0$ & $1.0$ \\
$TE_{Z \to Y}$ & ax=0 & ax=0.1 & $-1.1$ & $99$ & $0.274$ & $1.000$ \\
$TE_{Z \to Y}$ & ax=0 & ax=0.25 & $-2.349$ & $99$ & $0.021$ & $0.125$ \\
$TE_{Z \to Y}$ & ax=0 & ax=0.5 & $-2.355$ & $99$ & $0.020$ & $0.123$ \\
$TE_{Z \to Y}$ & ax=0.1 & ax=0.25 & $-1.577$ & $99$ & $0.118$ & $0.708$ \\
$TE_{Z \to Y}$ & ax=0.1 & ax=0.5 & $-1.586$ & $99$ & $0.116$ & $0.696$ \\
$TE_{Z \to Y}$ & ax=0.25 & ax=0.5 & $-0.312$ & $99$ & $0.756$ & $1.000$ \\
\bottomrule
\end{tabular}
\label{tab:trivariate-combined-te-ax}
\end{table}

\subsubsection{Effect on $CJTE_{(X,Z) \to Y|Z}$}
Comparing results on Figure \ref{fig:trivariate-cjte-xy-combined} it is visible that not only $a_x$ causes CJTE to increase, but $a_z$ too, what is confirmed in detail in Table \ref{tab:trivariate-cjte-xy}.

\begin{figure}[H]
    \centering
    \begin{subfigure}[b]{0.48\textwidth}
        \centering
        \includegraphics[width=\linewidth]{"figs/results/synthetic/Varying az Linear Trivariate cjte_(x,z)->y|z.png"}
        \caption{Varying common drive $a_z$}
        \label{fig:trivariate-cjte-x-y-az}
    \end{subfigure}
    \hfill
    \begin{subfigure}[b]{0.48\textwidth}
        \centering
        \includegraphics[width=\linewidth]{"figs/results/synthetic/Varying ax Linear Trivariate cjte_(x,z)->y|z.png"}
        \caption{Varying coupling $a_x$}
        \label{fig:trivariate-cjte-x-y-ax}
    \end{subfigure}
    
    \caption{CJTE results for the synthetic trivariate system along the $X \to Y$ information flow, conditioned on $Z$. (a) Results for varying common drive strength $a_z$; (b) Results for varying direct coupling strength $a_x$.}
    \label{fig:trivariate-cjte-xy-combined}
\end{figure}

\begin{table}[H]
\centering
\caption{Statistical analysis values for the CJTE results, $X \to Y$ (Figure \ref{fig:trivariate-cjte-xy-combined})}
\begin{tabular}{lllllll}
\toprule
Contrast & ddof1 & ddof2 & F & p-unc & ng2 & eps \\
\midrule
az & $3$ & $297$ & $122.339$ & $< 0.001$ & $0.391$ & $0.937$ \\
ax & $3$ & $297$ & $147.668$ & $< 0.001$ & $0.423$ & $0.906$ \\
\bottomrule
\end{tabular}
\vspace{0.5em}
\begin{tabular}{llllll}
\toprule
A & B & T & dof & p-unc & p-corr \\
\midrule
az=0 & az=0.1 & $-1.497$ & $99$ & $0.138$ & $0.826$ \\
az=0 & az=0.25 & $-10.441$ & $99$ & $< 0.001$ & $< 0.001$ \\
az=0 & az=0.5 & $-14.762$ & $99$ & $< 0.001$ & $< 0.001$ \\
az=0.1 & az=0.25 & $-9.356$ & $99$ & $< 0.001$ & $< 0.001$ \\
az=0.1 & az=0.5 & $-17.329$ & $99$ & $< 0.001$ & $< 0.001$ \\
az=0.25 & az=0.5 & $-6.287$ & $99$ & $< 0.001$ & $< 0.001$ \\
ax=0 & ax=0.1 & $-2.181$ & $99$ & $0.032$ & $0.189$ \\
ax=0 & ax=0.25 & $-6.803$ & $99$ & $< 0.001$ & $< 0.001$ \\
ax=0 & ax=0.5 & $-16.69$ & $99$ & $< 0.001$ & $< 0.001$ \\
ax=0.1 & ax=0.25 & $-5.473$ & $99$ & $< 0.001$ & $< 0.001$ \\
ax=0.1 & ax=0.5 & $-15.778$ & $99$ & $< 0.001$ & $< 0.001$ \\
ax=0.25 & ax=0.5 & $-13.146$ & $99$ & $< 0.001$ & $< 0.001$ \\
\bottomrule
\end{tabular}
\label{tab:trivariate-cjte-xy}
\end{table}

\subsubsection{Effect on $CJTE_{(Y,Z) \to X|Z}$}
For the opposite direction of information flow, CJTE is again increasing with increasing $a_z$, and is staying stable for varying $a_x$, what is noticable from Figure \ref{fig:trivariate-cjte-yx-combined}, and Table \ref{tab:trivariate-cjte-yx-combined}.

\begin{figure}[H]
    \centering
    \begin{subfigure}[b]{0.48\textwidth}
        \centering
        \includegraphics[width=\linewidth]{"figs/results/synthetic/Varying az Linear Trivariate cjte_(y,z)->x|z.png"}
        \caption{Varying common drive $a_z$}
        \label{fig:trivariate-cjte-y-x-az}
    \end{subfigure}
    \hfill
    \begin{subfigure}[b]{0.48\textwidth}
        \centering
        \includegraphics[width=\linewidth]{"figs/results/synthetic/Varying ax Linear Trivariate cjte_(y,z)->x|z.png"}
        \caption{Varying coupling $a_x$}
        \label{fig:trivariate-cjte-y-x-ax}
    \end{subfigure}
    
    \caption{CJTE results for the synthetic trivariate system against the $Y \to X$ information flow, conditioned on $Z$. (a) Results for varying common drive strength $a_z$; (b) Results for varying direct coupling strength $a_x$.}
    \label{fig:trivariate-cjte-yx-combined}
\end{figure}

\begin{table}[H]
\centering
\caption{Statistical analysis values for the Standard CJTE results, $Y \to X$ (Figure \ref{fig:trivariate-cjte-yx-combined})}
\begin{tabular}{lllllll}
\toprule
Contrast & ddof1 & ddof2 & F & p-unc & ng2 & eps \\
\midrule
az & $3$ & $297$ & $125.107$ & $< 0.001$ & $0.448$ & $0.888$ \\
ax & $3$ & $297$ & $1.795$ & $0.148$ & $0.006$ & $0.898$ \\
\bottomrule
\end{tabular}
\vspace{0.5em}
\begin{tabular}{llllll}
\toprule
A & B & T & dof & p-unc & p-corr \\
\midrule
az=0 & az=0.1 & $-1.789$ & $99$ & $0.077$ & $0.460$ \\
az=0 & az=0.25 & $-10.132$ & $99$ & $< 0.001$ & $< 0.001$ \\
az=0 & az=0.5 & $-14.376$ & $99$ & $< 0.001$ & $< 0.001$ \\
az=0.1 & az=0.25 & $-11.626$ & $99$ & $< 0.001$ & $< 0.001$ \\
az=0.1 & az=0.5 & $-16.067$ & $99$ & $< 0.001$ & $< 0.001$ \\
az=0.25 & az=0.5 & $-4.691$ & $99$ & $< 0.001$ & $< 0.001$ \\
ax=0 & ax=0.1 & $-0.853$ & $99$ & $0.396$ & $1.000$ \\
ax=0 & ax=0.25 & $-2.159$ & $99$ & $0.033$ & $0.200$ \\
ax=0 & ax=0.5 & $-0.529$ & $99$ & $0.598$ & $1.000$ \\
ax=0.1 & ax=0.25 & $-1.649$ & $99$ & $0.102$ & $0.613$ \\
ax=0.1 & ax=0.5 & $0.124$ & $99$ & $0.901$ & $1.000$ \\
ax=0.25 & ax=0.5 & $1.722$ & $99$ & $0.088$ & $0.529$ \\
\bottomrule
\end{tabular}
\label{tab:trivariate-cjte-yx-combined}
\end{table}

\subsubsection{Comparison of $TE_{X->Y}$ with $CJTE_{(X,Z)->Y|Z}$}\label{subsubsubsec:synthetic_comparison_cjte-te}
For non existing direct coupling from $X \to Y$ ($a_x=0$) CJTE yield statistically significant bigger mean value, but for stronger couplings ($a_x=0.25$, $a_x=0.5$) CJTE is significantly smaller than TE.

\begin{table}[H]
\centering
\caption{Statistical results of comparison between $TE_{X->Y}$ and $CJTE_{(X,Z)->Y|Z}$}
\begin{tabular}{lllllll}
\toprule
Condition & Mean te\_x->y & Mean cjte\_(x,z)->y|z & T & p-unc \\
\midrule
ax=0 & $0.104$ & $0.131$ & $-4.642$ & $< 0.001$ \\
ax=0.1 & $0.137$ & $0.143$ & $-1.103$ & $0.273$ \\
ax=0.25 & $0.201$ & $0.173$ & $5.259$ & $< 0.001$ \\
ax=0.5 & $0.31$ & $0.247$ & $10.025$ & $< 0.001$ \\
\bottomrule
\end{tabular}
\label{tab:comparison-trivariate-ax}
\end{table}

\section{Physiological Signals}

\subsection{BRS Estimation Using TE}\label{subsec:phys-te-result}
Figure \ref{fig:phys-te-results} shows that TE is the biggest for $6$ breaths/min, and is statistically different from baseline (i.e. breathing at rest) and $15$ breaths/min.
Additionally post hoc analysis from Table \ref{tab:phys-te-results} suggest significant difference between baseline and $10$ breath/min, and between $10$ and $15$ breaths/min.

\begin{figure}[H]
    \centering
    \includegraphics[width=0.9\linewidth]{figs/results/physiological/te_sap->hp.png}
    \caption{BRS estimation results using TE}
    \label{fig:phys-te-results}
\end{figure}

\begin{table}[H]
\centering
\caption{Statistical results of BRS estimation using TE}
\begin{tabular}{llllll}
\toprule
ddof1 & ddof2 & F & p-unc & ng2 & eps \\
\midrule
$3$ & $93$ & $15.726$ & $< 0.001$ & $0.21$ & $0.893$ \\
\bottomrule
\end{tabular}
\vspace{0.5em}
\begin{tabular}{llllll}
\toprule
A & B & T & dof & p-unc & p-corr \\
\midrule
CB BASELINE & CB 6 & $-5.544$ & $31$ & $< 0.001$ & $< 0.001$ \\
CB BASELINE & CB 10 & $-3.362$ & $31$ & $< 0.01$ & $0.012$ \\
CB BASELINE & CB 15 & $-0.332$ & $31$ & $0.742$ & $1.000$ \\
CB 6 & CB 10 & $2.76$ & $31$ & $< 0.01$ & $0.058$ \\
CB 6 & CB 15 & $4.817$ & $31$ & $< 0.001$ & $< 0.001$ \\
CB 10 & CB 15 & $3.32$ & $31$ & $< 0.01$ & $0.014$ \\
\bottomrule
\end{tabular}
\label{tab:phys-te-results}
\end{table}

\subsection{Information Flow From ETCO$_2$}\label{subsec:phys-etco-effect}
Combined results for both ETCO$_2$$\to$HP, and ETCO$_2$$\to$SAP in Figure \ref{fig:phys-etco2-combined}, and Table \ref{tab:phys-etco2-results} suggest that information conducted from ETCO$_2$ stays stable at non-zero level among all the groups.
The only statistically significant difference is only for flow to HP between baseline and $6$ breaths/min.

\begin{figure}[H]
    \centering
    \begin{subfigure}[b]{0.48\textwidth}
        \centering
        \includegraphics[width=\linewidth]{"figs/results/physiological/te_etco2->sap.png}
        \caption{Effect on SAP}
        \label{fig:phys-etco-sap}
    \end{subfigure}
    \hfill
    \begin{subfigure}[b]{0.48\textwidth}
        \centering
        \includegraphics[width=\linewidth]{"figs/results/physiological/te_etco2->hp.png}
        \caption{Effect on HP}
        \label{fig:phys-etco-hp}
    \end{subfigure}
    
    \caption{Common driver effect of ETCO$_2$ on (a) SAP; (b) HP}
    \label{fig:phys-etco2-combined}
\end{figure}

\begin{table}[H]
\centering
\caption{Statistical results of BRS estimation using TE}
\begin{tabular}{lllllll}
\toprule
Source & ddof1 & ddof2 & F & p-unc & ng2 & eps \\
\midrule
te\_etco2->sap & $3$ & $93$ & $0.407$ & $0.748$ & $0.01$ & $0.882$ \\
te\_etco2->hp & $3$ & $93$ & $2.607$ & $0.056$ & $0.045$ & $0.932$ \\
\bottomrule
\end{tabular}
\vspace{0.5em}
\begin{tabular}{lllllll}
\toprule
Source & A & B & T & dof & p-unc & p-corr \\
\midrule
te\_etco2->sap & CB BASELINE & CB 6 & $0.691$ & $31$ & $0.495$ & $1.000$ \\
te\_etco2->sap & CB BASELINE & CB 10 & $0.674$ & $31$ & $0.505$ & $1.000$ \\
te\_etco2->sap & CB BASELINE & CB 15 & $0.93$ & $31$ & $0.359$ & $1.000$ \\
te\_etco2->sap & CB 6 & CB 10 & $-0.023$ & $31$ & $0.982$ & $1.000$ \\
te\_etco2->sap & CB 6 & CB 15 & $0.39$ & $31$ & $0.699$ & $1.000$ \\
te\_etco2->sap & CB 10 & CB 15 & $0.358$ & $31$ & $0.723$ & $1.000$ \\
te\_etco2->hp & CB BASELINE & CB 6 & $2.872$ & $31$ & $< 0.01$ & $0.044$ \\
te\_etco2->hp & CB BASELINE & CB 10 & $2.031$ & $31$ & $0.051$ & $0.306$ \\
te\_etco2->hp & CB BASELINE & CB 15 & $1.856$ & $31$ & $0.073$ & $0.438$ \\
te\_etco2->hp & CB 6 & CB 10 & $-0.939$ & $31$ & $0.355$ & $1.000$ \\
te\_etco2->hp & CB 6 & CB 15 & $-0.692$ & $31$ & $0.494$ & $1.000$ \\
te\_etco2->hp & CB 10 & CB 15 & $0.253$ & $31$ & $0.802$ & $1.000$ \\
\bottomrule
\end{tabular}
\label{tab:phys-etco2-results}
\end{table}


\subsection{BRS Estimation Using CJTE}\label{subsec:phys-cjte-results}
The results are agregated in Table \ref{tab:phys-cjte-results} and plotted on Figure \ref{fig:phys-cjte-results}.
The only significant difference is between paced breathing at $6$ breaths/min and $15$ breaths/min, with shape simillar to one in Figure \ref{fig:phys-te-results}.

\begin{figure}[H]
    \centering
    \includegraphics[width=0.9\linewidth]{figs/results/physiological/cjte_(sap,etco2)->hp|etco2.png}
    \caption{BRS estimation results using CJTE, which conditions effect of ETCO$_2$}
    \label{fig:phys-cjte-results}
\end{figure}

\begin{table}[H]
\centering
\caption{Statistical results of BRS estimation using CJTE, with conditioning on ETCO$_2$}
\begin{tabular}{llllll}
\toprule
ddof1 & ddof2 & F & p-unc & ng2 & eps \\
\midrule
$3$ & $93$ & $4.32$ & $< 0.01$ & $0.069$ & $0.915$ \\
\bottomrule
\end{tabular}
\vspace{0.5em}
\begin{tabular}{llllll}
\toprule
A & B & T & dof & p-unc & p-corr \\
\midrule
CB BASELINE & CB 6 & $-2.435$ & $31$ & $0.021$ & $0.125$ \\
CB BASELINE & CB 10 & $-1.111$ & $31$ & $0.275$ & $1.000$ \\
CB BASELINE & CB 15 & $1.03$ & $31$ & $0.311$ & $1.000$ \\
CB 6 & CB 10 & $1.299$ & $31$ & $0.204$ & $1.000$ \\
CB 6 & CB 15 & $2.967$ & $31$ & $< 0.01$ & $0.035$ \\
CB 10 & CB 15 & $2.466$ & $31$ & $0.019$ & $0.116$ \\
\bottomrule
\end{tabular}
\label{tab:phys-cjte-results}
\end{table}

\subsection{Comparison of TE and CJTE in BRS estimation}\label{subsec:phys-comparison-results}
The final result from Table \ref{tab:phys-comparison-results} in this thesis, contains comparison of the mean values of TE and CJTE.
From pairwise comparisons it is shown that CJTE is significantly bigger for baseline, $10$ breaths/min, and $15$ breaths/min.

\begin{table}[H]
\centering
\caption{Comparison of TE and CJTE in BRS estimation}
\begin{tabular}{lllll}
\toprule
Condition & Mean te\_sap->hp & Mean cjte\_(sap,etco2)->hp|etco2 & T & p-unc \\
\midrule
CB BASELINE & $0.167$ & $0.297$ & $-8.401$ & $< 0.001$ \\
CB 6 & $0.342$ & $0.36$ & $-1.009$ & $0.321$ \\
CB 10 & $0.263$ & $0.325$ & $-3.769$ & $< 0.001$ \\
CB 15 & $0.175$ & $0.271$ & $-5.485$ & $< 0.001$ \\
\bottomrule
\end{tabular}
\label{tab:phys-comparison-results}
\end{table}
