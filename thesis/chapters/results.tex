\chapter{Results}
For a reliability all the statistical analysis were performed using \textit{pingouin python} library \cite{pingouin}.

Each analysis included calculation of the repeated measures analysis of variance (RM ANOVA), followed by the post hoc analysis using two-sided, pairwase, t-tests, with \textit{Bonferroni} $p$-value adjustments.

The following acronyms are used in the results tables, which are composed of the statistical calculation of the RM ANOVA (top panel) and the post hoc analysis (bottom panel):

\begin{itemize}
  \item \textbf{ddof1, ddof2}: Degrees of freedom for the RM ANOVA numerator and denominator, respectively.
  \item \textbf{F}: The F-statistic value from the RM ANOVA.
  \item \textbf{p-unc}: Uncorrected $p$-value (used in both ANOVA and post hoc tables).
  \item \textbf{$\eta^2_G$ (ng2)}: Generalized Eta Squared, representing the effect size for the RM ANOVA.
  \item \textbf{$\epsilon$ (eps)}: Epsilon, such as the Greenhouse–Geisser correction factor for sphericity violation.
  \item \textbf{T}: The T-statistic value from the pairwise t-tests.
  \item \textbf{dof}: Degrees of freedom for the pairwise t-tests.
  \item \textbf{p-corr}: Bonferroni-corrected $p$-value from the post hoc analysis.
\end{itemize}

\section{Synthetic Signals}

\subsection{Open-Loop Linear Causality with Varying Strength} 

\subsubsection{Causality Along the Information Flow ($X \to Y$)}
The results presented in Table \ref{tab:openloop-linear-a-along} indicate a statistically significant consistent increase between synthetic signals generated with varring $a_{21}$ value (standing for coupling strength from $X \to Y$) along the direction of real information flow ($X \to Y$), what can be observed in the figure \ref{fig:openloop-linear-a-along}.

\begin{figure}[H]
    \centering
    \includegraphics[width=0.9\linewidth]{figs/results/synthetic/rg_linear_openloop_a te_x->y.png}
    \caption{TE results for different a$_{21}$ values in lineary coupled openloop synthetic signal, along the real information flow direction}
    \label{fig:openloop-linear-a-along}
\end{figure}

\begin{table}[H]
\centering
\caption{Statistical analysis values for the Transfer Entropy results along the direction $X \to Y$ (Figure \ref{fig:openloop-linear-a-along})}
\begin{tabular}{llllll}
\toprule
ddof1 & ddof2 & F & p-unc & ng2 & eps \\
\midrule
$3$ & $138$ & $342.701$ & $< 0.001$ & $0.848$ & $0.793$ \\
\bottomrule
\end{tabular}
\vspace{0.5em}
\begin{tabular}{llllll}
\toprule
A & B & T & dof & p-unc & p-corr \\
\midrule
a21=0.05 & a21=0.25 & $-4.306$ & $46$ & $< 0.001$ & $< 0.001$ \\
a21=0.05 & a21=0.5 & $-14.664$ & $46$ & $< 0.001$ & $< 0.001$ \\
a21=0.05 & a21=1 & $-27.604$ & $46$ & $< 0.001$ & $< 0.001$ \\
a21=0.25 & a21=0.5 & $-8.49$ & $46$ & $< 0.001$ & $< 0.001$ \\
a21=0.25 & a21=1 & $-21.583$ & $46$ & $< 0.001$ & $< 0.001$ \\
a21=0.5 & a21=1 & $-15.588$ & $46$ & $< 0.001$ & $< 0.001$ \\
\bottomrule
\end{tabular}
\label{tab:openloop-linear-a-along}
\end{table}

\subsubsection{Causality Against the Information Flow ($Y \to X$)}
Conversely, when measuring Transfer Entropy in the direction against the real information flow ($Y \to X$), no statistically significant difference is observed across the varying across the varying $a_21$ values, as shown in Table \ref{tab:openloop-linear-a-against} and Figure \ref{fig:openloop-linear-a-against}.

\begin{figure}[H]
    \centering
    \includegraphics[width=0.9\linewidth]{figs/results/synthetic/rg_linear_openloop_a te_y->x.png}
    \caption{TE results for different a$_{21}$ values in lineary coupled openloop synthetic signal, against the real information flow direction}
    \label{fig:openloop-linear-a-against}
\end{figure}

\begin{table}[H]
\centering
\caption{Statistics analysis values for the figure \ref{fig:openloop-linear-a-against}}
\begin{tabular}{llllll}
\toprule
ddof1 & ddof2 & F & p-unc & ng2 & eps \\
\midrule
$3$ & $138$ & $0.776$ & $0.509$ & $0.012$ & $0.928$ \\
\bottomrule
\end{tabular}
\vspace{0.5em}
\begin{tabular}{llllll}
\toprule
A & B & T & dof & p-unc & p-corr \\
\midrule
a21=0.05 & a21=0.25 & $0.468$ & $46$ & $0.642$ & $1.000$ \\
a21=0.05 & a21=0.5 & $-1.132$ & $46$ & $0.263$ & $1.000$ \\
a21=0.05 & a21=1 & $0.21$ & $46$ & $0.834$ & $1.000$ \\
a21=0.25 & a21=0.5 & $-1.448$ & $46$ & $0.154$ & $0.926$ \\
a21=0.25 & a21=1 & $-0.193$ & $46$ & $0.848$ & $1.000$ \\
a21=0.5 & a21=1 & $1.054$ & $46$ & $0.297$ & $1.000$ \\
\bottomrule
\end{tabular}
\label{tab:openloop-linear-a-against}
\end{table}

\subsection{Open-Loop Linear Causality with Varying Signal Length}

\subsubsection{Causality Along the Information Flow ($X \to Y$)}
The results for the TE measured along the true causal direction ($X \to Y$) are presented in Figure \ref{fig:openloop-linear-l-along} and Table \ref{tab:openloop-linear-l-along}.

The RM ANOVA revealed a statistically significant effect of the signal length on the TE estimation ($F(3, 114) = 5.933, p < 0.001, \eta^2_G = 0.112$). 
As illustrated in Figure \ref{fig:openloop-linear-l-along}, the mean TE value systematically increases as the signal length is extended from $L=100$ to $L=500$. 
Post-hoc analysis confirms that the TE estimated with $L=100$ is significantly lower than that estimated with $L=500$ ($p < 0.01$). 
This trend is observed to level off, as no significant difference was found between the $L=500$ and $L=1000$ signal lengths.

\begin{figure}[H]
    \centering
    \includegraphics[width=0.9\linewidth]{figs/results/synthetic/linear_openloop_l te_x->y.png}
    \caption{TE results for varying signal length ($L$) in a linearly coupled open-loop synthetic signal, along the real information flow direction ($X \to Y$)}
    \label{fig:openloop-linear-l-along}
\end{figure}

\begin{table}[H]
\centering
\caption{Statistical analysis values for the Transfer Entropy results along the direction $X \to Y$ (Figure \ref{fig:openloop-linear-l-along})}
\begin{tabular}{llllll}
\toprule
ddof1 & ddof2 & F & p-unc & ng2 & eps \\
\midrule
$3$ & $114$ & $5.933$ & $< 0.001$ & $0.112$ & $0.735$ \\
\bottomrule
\end{tabular}
\vspace{0.5em}
\begin{tabular}{llllll}
\toprule
A & B & T & dof & p-unc & p-corr \\
\midrule
length=100 & length=200 & $-0.257$ & $38$ & $0.798$ & $1.000$ \\
length=100 & length=500 & $-3.43$ & $38$ & $< 0.01$ & $< 0.01$ \\
length=100 & length=1000 & $-2.697$ & $38$ & $0.010$ & $0.062$ \\
length=200 & length=500 & $-3.224$ & $38$ & $< 0.01$ & $0.016$ \\
length=200 & length=1000 & $-2.461$ & $38$ & $0.019$ & $0.111$ \\
length=500 & length=1000 & $1.483$ & $38$ & $0.146$ & $0.878$ \\
\bottomrule
\end{tabular}
\label{tab:openloop-linear-l-along}
\end{table}

\subsubsection{Causality Against the Information Flow ($Y \to X$)}
The results for the anti-causal direction ($Y \to X$) are shown in Figure \ref{fig:openloop-linear-l-against} and Table \ref{tab:openloop-linear-l-against}.

The RM ANOVA indicated a statistically significant difference across lengths ($F(3, 114) = 3.808, p = 0.012, \eta^2_G = 0.074$). 
However, the magnitude of the measured TE remains consistently close to zero across all lengths (Figure \ref{fig:openloop-linear-l-against}). 
Furthermore, the post-hoc analysis using Bonferroni correction showed no consistent or widespread significant differences between the length groups.

\begin{figure}[H]
    \centering
    \includegraphics[width=0.9\linewidth]{figs/results/synthetic/linear_openloop_l te_y->x.png}
    \caption{TE results for varying signal length ($L$) in a linearly coupled open-loop synthetic signal, against the real information flow direction ($Y \to X$)}
    \label{fig:openloop-linear-l-against}
\end{figure}

\begin{table}[H]
\centering
\caption{Statistical analysis values for the Transfer Entropy results against the direction $Y \to X$ (Figure \ref{fig:openloop-linear-l-against})}
\begin{tabular}{llllll}
\toprule
ddof1 & ddof2 & F & p-unc & ng2 & eps \\
\midrule
$3$ & $114$ & $3.808$ & $0.012$ & $0.074$ & $0.765$ \\
\bottomrule
\end{tabular}
\vspace{0.5em}
\begin{tabular}{llllll}
\toprule
A & B & T & dof & p-unc & p-corr \\
\midrule
length=100 & length=200 & $0.444$ & $38$ & $0.659$ & $1.000$ \\
length=100 & length=500 & $-2.23$ & $38$ & $0.032$ & $0.190$ \\
length=100 & length=1000 & $-1.42$ & $38$ & $0.164$ & $0.983$ \\
length=200 & length=500 & $-2.862$ & $38$ & $< 0.01$ & $0.041$ \\
length=200 & length=1000 & $-2.212$ & $38$ & $0.033$ & $0.198$ \\
length=500 & length=1000 & $1.935$ & $38$ & $0.060$ & $0.363$ \\
\bottomrule
\end{tabular}
\label{tab:openloop-linear-l-against}
\end{table}

\subsection{Open-Loop Linear Causality with Varying Noise Level}

\subsubsection{Causality Along the Information Flow ($X \to Y$)}
The results for the TE measured along the true causal direction ($X \to Y$) are presented in Figure \ref{fig:openloop-linear-e-along} and Table \ref{tab:openloop-linear-e-along}.

The RM ANOVA revealed no statistically significant effect of the noise standard deviation ($\epsilon$) on the TE estimation ($F(3, 141) = 0.141, p = 0.935, \eta^2_G = 0.002$). 
As illustrated in Figure \ref{fig:openloop-linear-e-along}, the mean TE values remain consistent across all tested noise levels. 
Furthermore, the post-hoc analysis confirms that no pairwise comparison between the $\epsilon$ groups showed a significant difference ($p\text{-corr}=1.000$ for all pairs).

\begin{figure}[H]
    \centering
    \includegraphics[width=0.9\linewidth]{figs/results/synthetic/rg_linear_openloop_e te_x->y.png}
    \caption{TE results for different noise standard deviation ($\epsilon$) values in a linearly coupled open-loop synthetic signal, along the real information flow direction ($X \to Y$)}
    \label{fig:openloop-linear-e-along}
\end{figure}

\begin{table}[H]
\centering
\caption{Statistical analysis values for the Transfer Entropy results along the direction $X \to Y$ (Figure \ref{fig:openloop-linear-e-along})}
\begin{tabular}{llllll}
\toprule
ddof1 & ddof2 & F & p-unc & ng2 & eps \\
\midrule
$3$ & $141$ & $0.141$ & $0.935$ & $0.002$ & $0.858$ \\
\bottomrule
\end{tabular}
\vspace{0.5em}
\begin{tabular}{llllll}
\toprule
A & B & T & dof & p-unc & p-corr \\
\midrule
epsilon=0.05 & epsilon=0.25 & $0.673$ & $47$ & $0.504$ & $1.000$ \\
epsilon=0.05 & epsilon=0.5 & $0.38$ & $47$ & $0.706$ & $1.000$ \\
epsilon=0.05 & epsilon=1 & $0.278$ & $47$ & $0.782$ & $1.000$ \\
epsilon=0.25 & epsilon=0.5 & $-0.327$ & $47$ & $0.745$ & $1.000$ \\
epsilon=0.25 & epsilon=1 & $-0.306$ & $47$ & $0.761$ & $1.000$ \\
epsilon=0.5 & epsilon=1 & $-0.01$ & $47$ & $0.992$ & $1.000$ \\
\bottomrule
\end{tabular}
\label{tab:openloop-linear-e-along}
\end{table}

\subsubsection{Causality Against the Information Flow ($Y \to X$)}

The results for the anti-causal direction ($Y \to X$) are presented in Figure \ref{fig:openloop-linear-e-against} and Table \ref{tab:openloop-linear-e-against}.

Similar to the causal direction, the RM ANOVA indicated no statistically significant difference in the TE values across the varying noise levels ($F(3, 141) = 0.332, p = 0.802, \eta^2_G = 0.006$). 
The measured TE remains consistently near zero, and the post-hoc analysis shows no significant pairwise differences.

\begin{figure}[H]
    \centering
    \includegraphics[width=0.9\linewidth]{figs/results/synthetic/rg_linear_openloop_e te_y->x.png}
    \caption{TE results for different noise standard deviation ($\epsilon$) values in a linearly coupled open-loop synthetic signal, against the real information flow direction ($Y \to X$)}
    \label{fig:openloop-linear-e-against}
\end{figure}

\begin{table}[H]
\centering
\caption{Statistical analysis values for the Transfer Entropy results against the direction $Y \to X$ (Figure \ref{fig:openloop-linear-e-against})}
\begin{tabular}{llllll}
\toprule
ddof1 & ddof2 & F & p-unc & ng2 & eps \\
\midrule
$3$ & $141$ & $0.332$ & $0.802$ & $0.006$ & $0.941$ \\
\bottomrule
\end{tabular}
\vspace{0.5em}
\begin{tabular}{llllll}
\toprule
A & B & T & dof & p-unc & p-corr \\
\midrule
epsilon=0.05 & epsilon=0.25 & $0.197$ & $47$ & $0.845$ & $1.000$ \\
epsilon=0.05 & epsilon=0.5 & $0.299$ & $47$ & $0.766$ & $1.000$ \\
epsilon=0.05 & epsilon=1 & $0.893$ & $47$ & $0.376$ & $1.000$ \\
epsilon=0.25 & epsilon=0.5 & $0.106$ & $47$ & $0.916$ & $1.000$ \\
epsilon=0.25 & epsilon=1 & $0.85$ & $47$ & $0.400$ & $1.000$ \\
epsilon=0.5 & epsilon=1 & $0.707$ & $47$ & $0.483$ & $1.000$ \\
\bottomrule
\end{tabular}
\label{tab:openloop-linear-e-against}
\end{table}

\subsection{Closed-Loop Linear Causality with Varying Strength}

\subsubsection{Causality Along the Set Information Flow ($X \to Y$)}

The results for the TE measured in the direction $X \to Y$ are presented in Figure \ref{fig:closedloop-linear-a-along} and Table \ref{tab:closedloop-linear-a-along}.

The RM ANOVA revealed a statistically significant effect of the coupling strength $a_{21}$ on the TE estimation ($F(3, 144) = 18.287, p < 0.001, \eta^2_G = 0.227$). 
As illustrated in Figure \ref{fig:closedloop-linear-a-along}, the mean TE value systematically increases with increasing $a_{21}$. 
Post-hoc analysis confirms that the TE estimated when $a_{21}=0$ is significantly lower than for any other tested coupling strength ($p < 0.01$ for $a_{21} \ge 0.1$).

\begin{figure}[H]
    \centering
    \includegraphics[width=0.9\linewidth]{figs/results/synthetic/rg_closed_loop te_x->y.png}
    \caption{TE results for different coupling strength ($a_{21}$) values in a linearly coupled closed-loop synthetic signal, in the $X \to Y$ direction}
    \label{fig:closedloop-linear-a-along}
\end{figure}

\begin{table}[H]
\centering
\caption{Statistical analysis values for the Transfer Entropy results in the direction $X \to Y$ (Figure \ref{fig:closedloop-linear-a-along})}
\begin{tabular}{llllll}
\toprule
ddof1 & ddof2 & F & p-unc & $\eta^2_G$ & $\epsilon$ \\
\midrule
$3$ & $144$ & $18.287$ & $< 0.001$ & $0.227$ & $0.930$ \\
\bottomrule
\end{tabular}
\vspace{0.5em}
\begin{tabular}{llllll}
\toprule
A & B & T & dof & p-unc & p-corr \\
\midrule
$a_{21}=0$ & $a_{21}=0.1$ & $-3.564$ & $48$ & $< 0.001$ & $< 0.01$ \\
$a_{21}=0$ & $a_{21}=0.15$ & $-4.813$ & $48$ & $< 0.001$ & $< 0.001$ \\
$a_{21}=0$ & $a_{21}=0.2$ & $-7.606$ & $48$ & $< 0.001$ & $< 0.001$ \\
$a_{21}=0.1$ & $a_{21}=0.15$ & $-1.669$ & $48$ & $0.102$ & $0.610$ \\
$a_{21}=0.1$ & $a_{21}=0.2$ & $-3.846$ & $48$ & $< 0.001$ & $< 0.01$ \\
$a_{21}=0.15$ & $a_{21}=0.2$ & $-2.318$ & $48$ & $0.025$ & $0.149$ \\
\bottomrule
\end{tabular}
\label{tab:closedloop-linear-a-along}
\end{table}

\subsubsection{Causality in the Reverse Direction ($Y \to X$)}

The results for the TE measured in the reverse direction $Y \to X$ are shown in Figure \ref{fig:closedloop-linear-a-against} and Table \ref{tab:closedloop-linear-a-against}.

The RM ANOVA indicated no statistically significant difference in the reverse TE across the varying $a_{21}$ values ($F(3, 144) = 2.035, p = 0.112, \eta^2_G = 0.032$). 
Although the post-hoc analysis shows some uncorrected $p$-values below the $0.05$ threshold (e.g., $a_{21}=0$ vs. $a_{21}=0.2$), all pairwise comparisons lost significance after the Bonferroni correction ($p$-corr $> 0.1$). 
The Transfer Entropy estimated in the reverse direction remains low across all coupling strengths.

\begin{figure}[H]
    \centering
    \includegraphics[width=0.9\linewidth]{figs/results/synthetic/rg_closed_loop te_y->x.png}
    \caption{TE results for different coupling strength ($a_{21}$) values in a linearly coupled closed-loop synthetic signal, in the $Y \to X$ direction}
    \label{fig:closedloop-linear-a-against}
\end{figure}

\begin{table}[H]
\centering
\caption{Statistical analysis values for the Transfer Entropy results in the direction $Y \to X$ (Figure \ref{fig:closedloop-linear-a-against})}
\begin{tabular}{llllll}
\toprule
ddof1 & ddof2 & F & p-unc & $\eta^2_G$ & $\epsilon$ \\
\midrule
$3$ & $144$ & $2.035$ & $0.112$ & $0.032$ & $0.945$ \\
\bottomrule
\end{tabular}
\vspace{0.5em}
\begin{tabular}{llllll}
\toprule
A & B & T & dof & p-unc & p-corr \\
\midrule
$a_{21}=0$ & $a_{21}=0.1$ & $-1.348$ & $48$ & $0.184$ & $1.000$ \\
$a_{21}=0$ & $a_{21}=0.15$ & $-2.144$ & $48$ & $0.037$ & $0.223$ \\
$a_{21}=0$ & $a_{21}=0.2$ & $-2.423$ & $48$ & $0.019$ & $0.115$ \\
$a_{21}=0.1$ & $a_{21}=0.15$ & $-0.47$ & $48$ & $0.640$ & $1.000$ \\
$a_{21}=0.1$ & $a_{21}=0.2$ & $-0.706$ & $48$ & $0.484$ & $1.000$ \\
$a_{21}=0.15$ & $a_{21}=0.2$ & $-0.332$ & $48$ & $0.742$ & $1.000$ \\
\bottomrule
\end{tabular}
\label{tab:closedloop-linear-a-against}
\end{table}

\subsection{Open-Loop Non-Linear Causality with Varying Strength}

\subsubsection{Causality Along the Information Flow ($X \to Y$)}

The results for the TE measured along the true causal direction ($X \to Y$) are presented in Figure \ref{fig:openloop-nonlinear-gamma-along} and Table \ref{tab:openloop-nonlinear-gamma-along}.

The RM ANOVA revealed a highly statistically significant effect of the non-linear coupling strength $\gamma$ on the TE estimation ($F(3, 144) = 168.819, p < 0.001, \eta^2_G = 0.727$). 
As illustrated in Figure \ref{fig:openloop-nonlinear-gamma-along}, the mean TE value increases substantially with increasing $\gamma$. P
ost-hoc analysis confirms that nearly all pairwise comparisons are highly significant ($p < 0.001$), demonstrating that the TE method successfully detects the increasing non-linear causal link as $\gamma$ increases.

\begin{figure}[H]
    \centering
    \includegraphics[width=0.9\linewidth]{figs/results/synthetic/rg_nonlinear te_x->y.png}
    \caption{TE results for different non-linear coupling strength ($\gamma$) values in an open-loop non-linear synthetic signal, along the real information flow direction ($X \to Y$)}
    \label{fig:openloop-nonlinear-gamma-along}
\end{figure}

\begin{table}[H]
\centering
\caption{Statistical analysis values for the Transfer Entropy results along the direction $X \to Y$ (Figure \ref{fig:openloop-nonlinear-gamma-along})}
\begin{tabular}{llllll}
\toprule
ddof1 & ddof2 & F & p-unc & ng2 & eps \\
\midrule
$3$ & $144$ & $168.819$ & $< 0.001$ & $0.727$ & $0.662$ \\
\bottomrule
\end{tabular}
\vspace{0.5em}
\begin{tabular}{llllll}
\toprule
A & B & T & dof & p-unc & p-corr \\
\midrule
gamma=0 & gamma=0.05 & $-1.733$ & $48$ & $0.089$ & $0.537$ \\
gamma=0 & gamma=0.1 & $-6.994$ & $48$ & $< 0.001$ & $< 0.001$ \\
gamma=0 & gamma=0.2 & $-15.948$ & $48$ & $< 0.001$ & $< 0.001$ \\
gamma=0.05 & gamma=0.1 & $-5.373$ & $48$ & $< 0.001$ & $< 0.001$ \\
gamma=0.05 & gamma=0.2 & $-15.539$ & $48$ & $< 0.001$ & $< 0.001$ \\
gamma=0.1 & gamma=0.2 & $-12.182$ & $48$ & $< 0.001$ & $< 0.001$ \\
\bottomrule
\end{tabular}
\label{tab:openloop-nonlinear-gamma-along}
\end{table}

\subsubsection{Causality Against the Information Flow ($Y \to X$)}

The results for the TE measured in the anti-causal direction ($Y \to X$) are presented in Figure \ref{fig:openloop-nonlinear-gamma-against} and Table \ref{tab:openloop-nonlinear-gamma-against}.

The RM ANOVA also indicated a statistically significant effect of $\gamma$ in the reverse direction ($F(3, 144) = 40.122, p < 0.001, \eta^2_G = 0.375$). 
Figure \ref{fig:openloop-nonlinear-gamma-against} shows that the TE value in the non-causal direction increases, particularly at $\gamma=0.2$. 
Post-hoc analysis confirms significant pairwise differences involving the highest coupling strength ($\gamma=0.2$ vs. $\gamma=0, \gamma=0.05, \gamma=0.1$, all $p < 0.001$). 
This result suggests that the non-linear coupling or the TE estimation for this non-linear model may introduce spurious non-causal influences at higher coupling strengths, a finding which warrants discussion.

\begin{figure}[H]
    \centering
    \includegraphics[width=0.9\linewidth]{figs/results/synthetic/rg_nonlinear te_y->x.png}
    \caption{TE results for different non-linear coupling strength ($\gamma$) values in an open-loop non-linear synthetic signal, against the real information flow direction ($Y \to X$)}
    \label{fig:openloop-nonlinear-gamma-against}
\end{figure}

\begin{table}[H]
\centering
\caption{Statistical analysis values for the Transfer Entropy results against the direction $Y \to X$ (Figure \ref{fig:openloop-nonlinear-gamma-against})}
\begin{tabular}{llllll}
\toprule
ddof1 & ddof2 & F & p-unc & ng2 & eps \\
\midrule
$3$ & $144$ & $40.122$ & $< 0.001$ & $0.375$ & $0.874$ \\
\bottomrule
\end{tabular}
\vspace{0.5em}
\begin{tabular}{llllll}
\toprule
A & B & T & dof & p-unc & p-corr \\
\midrule
gamma=0 & gamma=0.05 & $-0.811$ & $48$ & $0.422$ & $1.000$ \\
gamma=0 & gamma=0.1 & $-1.068$ & $48$ & $0.291$ & $1.000$ \\
gamma=0 & gamma=0.2 & $-8.328$ & $48$ & $< 0.001$ & $< 0.001$ \\
gamma=0.05 & gamma=0.1 & $-0.09$ & $48$ & $0.929$ & $1.000$ \\
gamma=0.05 & gamma=0.2 & $-8.806$ & $48$ & $< 0.001$ & $< 0.001$ \\
gamma=0.1 & gamma=0.2 & $-7.586$ & $48$ & $< 0.001$ & $< 0.001$ \\
\bottomrule
\end{tabular}
\label{tab:openloop-nonlinear-gamma-against}
\end{table}

\subsection{Open-Loop Linear Trivariate Causality with Varying Confounding Strength}

\subsubsection{TE}
The results for the standard TE from $X \to Y$, which does not account for the confounder $Z$, and the reverse direction are analyzed below.

\paragraph{TE: $X \to Y$}
The RM ANOVA revealed a statistically significant effect of the Z coupling strength on the TE estimation ($F(3, 135) = 33.627, p < 0.001, \eta^2_G = 0.360$). 
As shown in Figure \ref{fig:trivariate-te-x-y}, the TE value decreases significantly as the confounding coupling strength increases (see Table \ref{tab:trivariate-te-x-y} for post-hoc details).

\begin{figure}[H]
    \centering
    \includegraphics[width=0.9\linewidth]{figs/results/synthetic/trivariate te_x->y.png}
    \caption{Standard TE results for the $X \to Y$ link as a function of the confounding variable $Z$ coupling strength (Z coupling)}
    \label{fig:trivariate-te-x-y}
\end{figure}

\begin{table}[H]
\centering
\caption{Statistical analysis values for the Standard TE results, $X \to Y$ (Figure \ref{fig:trivariate-te-x-y})}
\begin{tabular}{llllll}
\toprule
ddof1 & ddof2 & F & p-unc & ng2 & eps \\
\midrule
$3$ & $135$ & $33.627$ & $< 0.001$ & $0.36$ & $0.947$ \\
\bottomrule
\end{tabular}
\vspace{0.5em}
\begin{tabular}{llllll}
\toprule
A & B & T & dof & p-unc & p-corr \\
\midrule
Z coupling=0.0 & Z coupling=0.25 & $-0.533$ & $45$ & $0.596$ & $1.000$ \\
Z coupling=0.0 & Z coupling=0.5 & $-5.981$ & $45$ & $< 0.001$ & $< 0.001$ \\
Z coupling=0.0 & Z coupling=1 & $-9.466$ & $45$ & $< 0.001$ & $< 0.001$ \\
Z coupling=0.25 & Z coupling=0.5 & $-4.295$ & $45$ & $< 0.001$ & $< 0.001$ \\
Z coupling=0.25 & Z coupling=1 & $-7.601$ & $45$ & $< 0.001$ & $< 0.001$ \\
Z coupling=0.5 & Z coupling=1 & $-3.344$ & $45$ & $< 0.01$ & $0.010$ \\
\bottomrule
\end{tabular}
\label{tab:trivariate-te-x-y}
\end{table}

\paragraph{TE: $Y \to X$}
The RM ANOVA for the anti-causal direction also indicated a statistically significant effect ($F(3, 135) = 28.375, p < 0.001, \eta^2_G = 0.330$). 
Figure \ref{fig:trivariate-te-y-x} shows that the TE value increases significantly in the reverse direction as the confounding coupling strength increases. 
This false positive result indicates that the standard TE is highly susceptible to the presence of an unobserved common driver (confounding variable).

\begin{figure}[H]
    \centering
    \includegraphics[width=0.9\linewidth]{figs/results/synthetic/trivariate te_y->x.png}
    \caption{Standard TE results for the $Y \to X$ link as a function of the confounding variable $Z$ coupling strength (Z coupling)}
    \label{fig:trivariate-te-y-x}
\end{figure}

\begin{table}[H]
\centering
\caption{Statistical analysis values for the Standard TE results, $Y \to X$ (Figure \ref{fig:trivariate-te-y-x})}
\begin{tabular}{llllll}
\toprule
ddof1 & ddof2 & F & p-unc & ng2 & eps \\
\midrule
$3$ & $135$ & $28.375$ & $< 0.001$ & $0.33$ & $0.943$ \\
\bottomrule
\end{tabular}
\vspace{0.5em}
\begin{tabular}{llllll}
\toprule
A & B & T & dof & p-unc & p-corr \\
\midrule
Z coupling=0.0 & Z coupling=0.25 & $-1.757$ & $45$ & $0.086$ & $0.515$ \\
Z coupling=0.0 & Z coupling=0.5 & $-4.884$ & $45$ & $< 0.001$ & $< 0.001$ \\
Z coupling=0.0 & Z coupling=1 & $-8.52$ & $45$ & $< 0.001$ & $< 0.001$ \\
Z coupling=0.25 & Z coupling=0.5 & $-3.407$ & $45$ & $< 0.01$ & $< 0.01$ \\
Z coupling=0.25 & Z coupling=1 & $-7.07$ & $45$ & $< 0.001$ & $< 0.001$ \\
Z coupling=0.5 & Z coupling=1 & $-3.041$ & $45$ & $< 0.01$ & $0.024$ \\
\bottomrule
\end{tabular}
\label{tab:trivariate-te-y-x}
\end{table}

\subsubsection{CJTE}

\begin{figure}[H]
    \centering
    \includegraphics[width=0.9\linewidth]{figs/results/synthetic/trivariate cjte_(x,z)->y|z.png}
    \caption{}
\end{figure}

\begin{table}[H]
\centering
\caption{}
\begin{tabular}{llllll}
\toprule
ddof1 & ddof2 & F & p-unc & ng2 & eps \\
\midrule
$3$ & $135$ & $16.054$ & $< 0.001$ & $0.205$ & $0.925$ \\
\bottomrule
\end{tabular}
\vspace{0.5em}
\begin{tabular}{llllll}
\toprule
A & B & T & dof & p-unc & p-corr \\
\midrule
z coupling=0.0 & z coupling=0.25 & $-1.343$ & $45$ & $0.186$ & $1.000$ \\
z coupling=0.0 & z coupling=0.5 & $-5.02$ & $45$ & $< 0.001$ & $< 0.001$ \\
z coupling=0.0 & z coupling=1 & $-7.013$ & $45$ & $< 0.001$ & $< 0.001$ \\
z coupling=0.25 & z coupling=0.5 & $-2.91$ & $45$ & $< 0.01$ & $0.034$ \\
z coupling=0.25 & z coupling=1 & $-4.47$ & $45$ & $< 0.001$ & $< 0.001$ \\
z coupling=0.5 & z coupling=1 & $-1.524$ & $45$ & $0.135$ & $0.807$ \\
\bottomrule
\end{tabular}
\end{table}

\begin{figure}[H]
    \centering
    \includegraphics[width=0.9\linewidth]{figs/results/synthetic/trivariate cjte_(y,z)->x|z.png}
    \caption{}
\end{figure}

\begin{table}[H]
\centering
\caption{}
\begin{tabular}{llllllllll}
\toprule
ddof1 & ddof2 & F & p-unc & ng2 & eps \\
\midrule
$3$ & $135$ & $34.52$ & $< 0.001$ & $0.368$ & $0.843$ \\
\bottomrule
\end{tabular}
\vspace{0.5em}
\begin{tabular}{llllll}
\toprule
A & B & T & dof & p-unc & p-corr \\
\midrule
z coupling=0.0 & z coupling=0.25 & $-1.548$ & $45$ & $0.129$ & $0.772$ \\
z coupling=0.0 & z coupling=0.5 & $-6.794$ & $45$ & $< 0.001$ & $< 0.001$ \\
z coupling=0.0 & z coupling=1 & $-9.039$ & $45$ & $< 0.001$ & $< 0.001$ \\
z coupling=0.25 & z coupling=0.5 & $-3.417$ & $45$ & $< 0.01$ & $< 0.01$ \\
z coupling=0.25 & z coupling=1 & $-8.777$ & $45$ & $< 0.001$ & $< 0.001$ \\
z coupling=0.5 & z coupling=1 & $-3.582$ & $45$ & $< 0.001$ & $< 0.01$ \\
\bottomrule
\end{tabular}
\end{table}

\section{Physiological Signals}

\subsection{TE}
\begin{figure}[H]
    \centering
    \includegraphics[width=0.9\linewidth]{figs/results/physiological/te_sap->hp.png}
    \caption{}
\end{figure}

\begin{table}[H]
\centering
\caption{}
\begin{tabular}{llllll}
\toprule
ddof1 & ddof2 & F & p-unc & ng2 & eps \\
\midrule
$3$ & $87$ & $14.123$ & $< 0.001$ & $0.211$ & $0.853$ \\
\bottomrule
\end{tabular}
\vspace{0.5em}
\begin{tabular}{llllll}
\toprule
A & B & T & dof & p-unc & p-corr \\
\midrule
CB BASELINE & CB 6 & $-5.14$ & $29$ & $< 0.001$ & $< 0.001$ \\
CB BASELINE & CB 10 & $-3.405$ & $29$ & $< 0.01$ & $0.012$ \\
CB BASELINE & CB 15 & $-0.377$ & $29$ & $0.709$ & $1.000$ \\
CB 6 & CB 10 & $2.268$ & $29$ & $0.031$ & $0.186$ \\
CB 6 & CB 15 & $4.451$ & $29$ & $< 0.001$ & $< 0.001$ \\
CB 10 & CB 15 & $3.476$ & $29$ & $< 0.01$ & $< 0.01$ \\
\bottomrule
\end{tabular}
\end{table}

\subsection{CJTE}
\begin{figure}[H]
    \centering
    \includegraphics[width=0.9\linewidth]{figs/results/physiological/cjte_(sap,etco2)->hp|etco2.png}
    \caption{CJTE results for $SAP \to HP$ information transfer that includes confounding effect of $ETCO_2$, at different breathing pace}
\end{figure}

\begin{table}[H]
\centering
\caption{}
\begin{tabular}{llllll}
\toprule
ddof1 & ddof2 & F & p-unc & ng2 & eps \\
\midrule
$3$ & $87$ & $4.871$ & $< 0.01$ & $0.089$ & $0.908$ \\
\bottomrule
\end{tabular}
\vspace{0.5em}
\begin{tabular}{llllll}
\toprule
A & B & T & dof & p-unc & p-corr \\
\midrule
CB BASELINE & CB 6 & $-2.7$ & $29$ & $0.011$ & $0.069$ \\
CB BASELINE & CB 10 & $-2.026$ & $29$ & $0.052$ & $0.313$ \\
CB BASELINE & CB 15 & $0.562$ & $29$ & $0.578$ & $1.000$ \\
CB 6 & CB 10 & $0.657$ & $29$ & $0.517$ & $1.000$ \\
CB 6 & CB 15 & $2.79$ & $29$ & $< 0.01$ & $0.055$ \\
CB 10 & CB 15 & $2.958$ & $29$ & $< 0.01$ & $0.037$ \\
\bottomrule
\end{tabular}
\end{table}
